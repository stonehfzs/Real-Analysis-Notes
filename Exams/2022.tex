\documentclass[UTF8]{article}

\usepackage{amsmath}%数学公式宏包
\usepackage{amssymb}%数学符号宏包
\usepackage{amsthm}%定理证明环境宏包
\usepackage{epstopdf}%eps转pdf宏包
\usepackage{fancyhdr}%页眉页脚宏包
\usepackage{gensymb}%包含\celsius,\degree,\micro,\ohm,\perthousand等命令
\usepackage{geometry}%页面设置宏包
\usepackage{graphicx}%插入图片宏包
\usepackage{pgfplots}%绘制图像宏包
\usepackage{relsize}
\usepackage{setspace}%设置间距宏包
\usepackage{subfigure}%子图排列宏包
\usepackage{tikz}%绘制图像宏包
\usepackage[colorlinks,linkcolor=blue]{hyperref}%超链接宏包
\usepackage[heading]{ctex}

%选择题分行命令
\newcommand{\onech}[4]{\indent\makebox[92pt][l]{\qquad A. #1} \hfill\makebox[92pt][l]{B. #2} \hfill \makebox[92pt][l]{C. #3} \hfill \makebox[92pt][l]{D. #4}\\}
\newcommand{\twoch}[4]{	\indent\makebox[110pt][l]{\qquad A. #1} \hfill\makebox[220pt][l]{B. #2}\\\indent\makebox[110pt][l]{\qquad C. #3} \hfill\makebox[220pt][l]{D. #4}\\}
\newcommand{\fourch}[4]{\indent\makebox[262pt][l]{\qquad A. #1}\\\indent\makebox[262pt][l]{\qquad B. #2}\\\indent\makebox[262pt][l]{\qquad C. #3}\\\indent\makebox[262pt][l]{\qquad D. #4}\\}
%罗马数字命令
\newcommand{\RNum}[1]{\uppercase\expandafter{\romannumeral #1\relax}}

%页面设置
\geometry{a4paper,left=2cm,right=2cm,top=2cm,bottom=2cm}
%首行缩进设置
\setlength{\parindent}{0em}
%行距设置
\setlength{\baselineskip}{20pt}
\linespread{1.75}
%页眉页脚设置 
\pagestyle{fancy}
\lhead{}
\chead{}
\rhead{\textbf{\rightmark} \qquad \thepage}
\lfoot{}
\cfoot{}
\rfoot{}
\renewcommand{\headrulewidth}{0pt}
\renewcommand{\footrulewidth}{0pt}

%定理证明环境设置
\newtheorem*{theorem}{Thm}
\newtheorem*{corollary}{Cor}
\newtheorem*{lemma}{Lem}
\newtheorem*{definition}{Def}
\newtheorem*{proposition}{Prop}
\newtheorem*{remark}{Rem}
\newtheorem{example}{Example}
\newtheorem{solution}{Sol}
\newtheorem{num}{}

\begin{document}
  \begin{center}
    \LARGE \textbf{2022春-实变函数}
  \end{center}

  \subsection*{一、填空题}

  1. 设$A_{2n-1}=\left(0,\frac{1}{n}\right),A_{2n}=\left(0,2n\right)$, 则$\limsup \limits_{n \to +\infty}A_n=$\underline{\hspace{1cm}}, $\liminf \limits_{n \to +\infty}A_n=$\underline{\hspace{1cm}}.\par

  2. 设$C$为Cantor集, 则$m(C)=$\underline{\hspace{1cm}}, 闭包$\overline{C}=$\underline{\hspace{1cm}}.\par

  3. 设$E\subset \mathbb{R}^n$, 若\underline{\hspace{1cm}}, 则称$E$是闭集. 此外闭集的等价条件还有:\underline{\hspace{1cm}}.\par

  4. 设$\{f_n(x)\}$是可测集$E$上几乎处处有限的实函数, 则$E\left[f_n\rightarrow 0\right]$=\underline{\hspace{1cm}}.\par

  \subsection*{二、选择题}

  1. 设$f:\mathbb{R}^n\rightarrow \mathbb{R}$, $A,B\subset \mathbb{R}^n$, 则下列集合关系成立的是:\par
  \twoch{$f(A\cup B)=f(A)\cup f(B)$}{$(A\backslash B)\cup B=A$}{$(B\backslash A)\cup A\subset A$}{$f(A\cap B)=f(A)\cap f(B)$}
    
  2. 设$E=\left\{(x,y)\in \mathbb{R}^2:0<x<1, y \in \mathbb{Q}\right\}$, 则下列命题成立的的是:\par
  \twoch{$m(E)=1$}{$m(E)=0$}{$E$是$\mathbb{R}^2$中闭集}{$E$是$\mathbb{R}^2$中开集}

  3. 设$f(x)$是可测集$E$上的实函数, 则下列命题不成立的是:\par
  \fourch{$f(x)$在$E$上可测当且仅当$|f(x)|$在$E$上可测}{若$f(x)$在$E$上可测, 则$f(x)$在$E$的任意子集上可测}{若$f(x)$在$E$上可测, 则$f(x)$在$E$的任意测度为零的子集上可积}{若$f(x)$在$E$上可测, 则$f(x)$在$E$上几乎处处有限}

  4. 设$f_k,f$均为可测集$E$上的可测函数, 当$k \to \infty$时下列命题成立的是:\par
  \fourch{若$f_k$在$E$上几乎处处收敛到$f$, 则$f_k$在$E$上近乎一致收敛到$E$.}{若$f_k$在$E$上几乎处处收敛到$f$, 则$\lim \limits_{k \to \infty}\int_{E}f_k(x)\mathrm{d}x=\int \limits_{E}f(x)\mathrm{d}x$.}{若$f_k$在$E$上近乎一致收敛到$f$, 则$f_k$在$E$上依测度收敛到$f$.}{若$f_k$在$E$上依测度收敛到$f$, 则$\lim \limits_{k \to \infty}\int_{E}f_k(x)\mathrm{d}x=\int_{E}f(x)\mathrm{d}x$.}
  
  5. 下列命题成立的是:\par
  \fourch{若$E\subset \mathbb{R}^n$测度为0, 则$E$为至多可数集.}{设$E\subset \mathbb{R}^n$, 存在包含$E$的开集$G$, $m(G\backslash E)=0$}{若$G\subset \mathbb{R}^n$是开集, 则$\partial G$是零测集.}{零测集是可测集.}

  \subsection*{三、解答题}

  1. 定义函数$f(x)=\begin{cases}
    x^2, & x\in \mathbb{Q} \\
    1+x, & x\in \mathbb{R}\backslash \mathbb{Q}
  \end{cases}$, 回答下列问题:\par
  (1) $f(x)$在$[0,1]$上是否$R$可积? 简单说明理由.\par
  (2) $f(x)$在$[0,1]$上是否$L$可积? 简单说明理由.\par

  2. 设$f$为可测集$E$上的可积函数, 记$E_k=E\left[|f|<\frac{1}{k}\right]$, 计算$\lim \limits_{k \to +\infty}\int_{E_k}|f(x)|\mathrm{d}x$.\par

  \subsection*{四、证明题}

  1. 设$f(x)$是$E$上的可测函数, 证明$\forall a >0, m(E\left[f\geqslant a\right])\leqslant \mathrm{e}^{-a}\int_{E}\mathrm{e}^{f(x)}\mathrm{d}x$.\par

  2. 设$f(x)$是$E$上的可积函数, $\{A_k\}$是$E$的一列可测子集且$\lim \limits_{k \to \infty}m(A_k)=0$, 证明$\lim \limits_{k \to \infty}\int_{A_k}f(x)\mathrm{d}x=0$.\par

  3. 设$f(x)$是$E$上的可积函数, 且对任意可测集$A\subset E$, 均有$\int_{A}f(x)\mathrm{d}x=0$. 证明$f(x)=0$ a.e. on $E$.\par

  \subsection*{五、辨析题}

  设$\{f_n\}$是可测集$E$上的非负可测函数列, 且$\{f_n\}$在$E$上几乎处处收敛到$0$.\par
  (1) 判断$\{f_n\}$在$E$上是否依测度收敛到$0$, 并简要说明理由.\par
  (2) 若$\int_{E} \max{f_1(x),f_2(x),\ldots,f_n(x)}\mathrm{d}x\leqslant M$, 验证$\lim \limits_{n \to \infty}\int_{E}f_n(x)\mathrm{d}x=0$.\par
  (3) 当$\lim \limits_{n \to \infty}\int_{E}f_n(x)\mathrm{d}x=0$时判断$\{f_n\}$在$E$上是否依测度收敛到$0$, 并简要说明理由.\par

\end{document}