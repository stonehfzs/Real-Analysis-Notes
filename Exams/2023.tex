\documentclass[UTF8]{article}

\usepackage{amsmath}%数学公式宏包
\usepackage{amssymb}%数学符号宏包
\usepackage{amsthm}%定理证明环境宏包
\usepackage{epstopdf}%eps转pdf宏包
\usepackage{fancyhdr}%页眉页脚宏包
\usepackage{gensymb}%包含\celsius,\degree,\micro,\ohm,\perthousand等命令
\usepackage{geometry}%页面设置宏包
\usepackage{graphicx}%插入图片宏包
\usepackage{pgfplots}%绘制图像宏包
\usepackage{relsize}
\usepackage{setspace}%设置间距宏包
\usepackage{subfigure}%子图排列宏包
\usepackage{tikz}%绘制图像宏包
\usepackage[colorlinks,linkcolor=blue]{hyperref}%超链接宏包
\usepackage[heading]{ctex}

%选择题分行命令
\newcommand{\onech}[4]{\indent\makebox[92pt][l]{\qquad A. #1} \hfill\makebox[92pt][l]{B. #2} \hfill \makebox[92pt][l]{C. #3} \hfill \makebox[92pt][l]{D. #4}\\}
\newcommand{\twoch}[4]{	\indent\makebox[110pt][l]{\qquad A. #1} \hfill\makebox[220pt][l]{B. #2}\\\indent\makebox[110pt][l]{\qquad C. #3} \hfill\makebox[220pt][l]{D. #4}\\}
\newcommand{\fourch}[4]{\indent\makebox[262pt][l]{\qquad A. #1}\\\indent\makebox[262pt][l]{\qquad B. #2}\\\indent\makebox[262pt][l]{\qquad C. #3}\\\indent\makebox[262pt][l]{\qquad D. #4}\\}
%罗马数字命令
\newcommand{\RNum}[1]{\uppercase\expandafter{\romannumeral #1\relax}}

%页面设置
\geometry{a4paper,left=2cm,right=2cm,top=2cm,bottom=2cm}
%首行缩进设置
\setlength{\parindent}{0em}
%行距设置
\setlength{\baselineskip}{20pt}
\linespread{1.75}
%页眉页脚设置 
\pagestyle{fancy}
\lhead{}
\chead{}
\rhead{\textbf{\rightmark} \qquad \thepage}
\lfoot{}
\cfoot{}
\rfoot{}
\renewcommand{\headrulewidth}{0pt}
\renewcommand{\footrulewidth}{0pt}

%定理证明环境设置
\newtheorem*{theorem}{Thm}
\newtheorem*{corollary}{Cor}
\newtheorem*{lemma}{Lem}
\newtheorem*{definition}{Def}
\newtheorem*{proposition}{Prop}
\newtheorem*{remark}{Rem}
\newtheorem{example}{Example}
\newtheorem{solution}{Sol}
\newtheorem{num}{}

\begin{document}
  \begin{center}
    \LARGE \textbf{2023春-实变函数}
  \end{center}

  \subsection*{一、选择题}

  1. 设$E=[a,b]\times \mathbb{Q} \times \mathbb{Q}$, 其中$\mathbb{Q}$为有理数集, 则$E$的讨论为真的是:\par
  \twoch{$m(E)=+\infty$.}{$m(E)=0$.}{$m(E)=b-a$.}{$E$是不可测集.}

  2. 设$f$为可测集$E$上的可测函数, 则关于$f$的命题成立的是:\par
  \fourch{$f$在$E$上Lebesgue可积当且仅当$f^{+}$和$f^{-}$在$E$上Lebesgue可积.}{$f$在$E$上Lebesgue可积当且仅当$|f|$在$E$上Lebesgue可积.}{$f$在$E$上Riemann可积当且仅当$|f|$在$E$上Riemann可积.}{$f$可测当且仅当$|f|$可测.}

  3. 设$A_n=\left[-\frac{1}{n}\right.,\left.1+(-1)^n\cdot \frac{1}{2n}\right)$, 则下列关于$A_n$的运算正确的是:\par
  \twoch{$\lim \limits_{n\to +\infty}A_n=[0,1)$}{$\liminf \limits_{n\to +\infty}A_n=[0,1]$}{$\limsup \limits_{n\to +\infty}A_n=[0,1)$}{$\limsup \limits_{n\to +\infty}A_n=[0,1]$}

  4. 设$C$为Cantor集, 则下列关于$C$的命题不成立的是:\par
  \twoch{$m(C)=0$.}{$\overline{C}=C$.}{$C$中有至多可数个点.}{$C$是疏朗集.}
  
  5. 设$E\subset \mathbb{R}^n$是Lebesgue可测集, 则下列关于$E$的命题成立的是:\par
  \fourch{$\forall \varepsilon >0$, 存在紧集$K\subset E$, $m(E\backslash K)<\varepsilon$.}{$\forall \varepsilon >0$, 存在闭集$F\subset E$, $m(E\backslash F)<\varepsilon$.}{$\forall \varepsilon >0$, 存在紧集$K\subset E$, $m(E)<m(K)+\varepsilon$.}{$\forall \varepsilon >0$, 存在闭集$F\subset E$, $m(E)<m(F)+\varepsilon$.}

  \subsection*{二、判断题}

  1. 若$E\subset \mathbb{R}^n$是零测集, 则$E$的闭包也是零测集.\par

  2. 若$\{f_n\},\{g_n\}$是可测集$E$上的几乎处处有限的可测函数, 若$f_n\Rightarrow f$, $g_n \Rightarrow g$, 则$f_ng_n\Rightarrow fg$.\par

  3. 设$E_k$为一列可测集, 若$E=\bigcap_{k=1}^{+\infty}E_k$, 则$m(E)=\lim \limits_{k \to +\infty}m(E_k)$.\par

  4. 设$A$为集合, 则总存在集合$B$, $A$的势严格大于$B$的势.\par

  5. 设$f_n(x)$为可测集$E$上几乎处处有限的实函数, 则$E\left[f_n\nrightarrow 0\right]=\bigcup_{k\geqslant 1} \bigcap_{N\geqslant 1}\bigcup_{n\geqslant N} E\left[|f_n(x)|\geqslant \frac{1}{k}\right]$.\par

  6. 有限个闭集的并是闭集, 可数多个闭集的并不一定是闭集.\par

  \subsection*{三、简答题}

  1. 设$e_n$是一列可测集, 分析$\limsup \limits_{n \ to +\infty}e_n$是否可测.\par

  2. 叙述几乎处处收敛和依测度收敛的定义, 并分析两者的关系.\par

  3. 设$C$为Cantor集, 设$f(x)=\begin{cases}
    \frac{\ln x}{x}, & x\in C \\
    \mathrm{e}^x+1, & x\notin C
  \end{cases}$.\par
  (1) $f$是否在$[0,1]$上Riemann可积? 简单说明理由.\par
  (2) $f$是否在$[0,1]$上Lebesgue可积? 简单说明理由.\par

  \subsection*{四、证明题}

  1. 设$f$为可测集$E$上的可积函数, 记$E_k=E\left[|f|<\frac{1}{k^2}\right]$, 证明$\lim_{k \to +\infty}\int_{E_k}|f(x)|\mathrm{d}x=0$.\par

  2. 设$f_k(x)$为可测集$E$上的可测函数列, 若$\lim_{k \to +\infty}\int_{E_k}|f(x)|\mathrm{d}x=0$, 证明$f_k(x)$在$E$上依测度收敛到0.\par

  \subsection*{五、解答题}

  1. 设$e_k$是一列集合, 若$m^{*}(e_k)<\frac{1}{2^k}$, 计算$\limsup \limits_{k \to +\infty}e_k$的外测度.\par

  2. 计算$\lim \limits_{n \to +\infty}\int_{0}^{1}\frac{kx \sin{kx}}{2+k^2x^3}\mathrm{d}x$.\par

\end{document}