% \documentclass[lang=cn,newtx,10pt,scheme=chinese]{elegantbook}
\documentclass[theorem=false,mathfont=none,openany,sub3section]{easybook}

\usepackage[lang=cn]{eb-elegantbook}
\usepackage{lmodern}
\usepackage{codehigh}
\lstset{moreemph={nofont,thmenv}}

\renewcommand{\rmdefault}{lmr}
\renewcommand{\sfdefault}{lmss}
\renewcommand{\ttdefault}{lmtt}

\title{Real Analysis}
\subtitle{Lecture Notes}
\author{Stone Sun}
%\institute{Ocean University of China}
\date{\today}
%\version{}
\bioinfo{联系方式}{hefengzhishui@outlook.com}

%\extrainfo{注意:Elegant系列模板自 2023 年 1 月 1 日开始,不再更新和维护!}

\setcounter{tocdepth}{3}

\logo{logo-blue.png}
\cover{cover.jpg}

% 本文档命令
% \usepackage{array}
\newcommand{\ccr}[1]{\makecell{{\color{#1}\rule{1cm}{1cm}}}}

% 修改标题页的橙色带
\definecolor{customcolor}{RGB}{32,178,170}
\colorlet{coverlinecolor}{customcolor}
% \usepackage{cprotect}

\addbibresource[location=local]{reference.bib} % 参考文献,不要删除

% 定义\btocgroup和\etocgroup命令把目录风格限制在组内,使其局部生效
\newcommand{\btocgroup}[1][toc]{\addtocontents{#1}{\string\begingroup}}
\newcommand{\etocgroup}[1][toc]{\addtocontents{#1}{\string\endgroup}}
%\SetTocStyle{chapter}{emph}{tocformat+ = \color{black}}
\let\ls\lstinline
\ebhdrset{footnotetype=flush}
\ctexset{
  paragraph/numbering=false,
  paragraph/beforeskip=1explus.2ex
  }
\SetTocStyle{subsubsection}{sub2}{
  tocindent=3.8em,
  tocformat+=\color{blue}
  }
\UseTocStyle{subsubsection}{sub2}{toc}

\begin{document}

\maketitle
\begin{center}
谨以此篇, 献给热爱分析的你.\par
\end{center}
\frontmatter

\begingroup
\renewcommand{\familydefault}{\rmdefault}
\tableofcontents
\endgroup

\newpage
\begin{center}
\Large
\textbf{前言}\par
\end{center}

\hspace{2em}
这是一份关于实分析(又名实变函数)的讲义, 主要涵盖了$\mathbb{R}^n$上的集合论与测度论, 同时讨论了Lebesgue可测、Lebesgue积分的基本概念和定理. 这份讲义是基于中国海洋大学的实变函数课程的讲义和笔记而写成的, 也参考了其他一些教材和讲义.\par
值得注意的是, 这门课程在不同开课学院的所占学分和所需学时是不同的. 所以这份讲义可能更适合每周3学时的同学学习参考.\par
这份讲义的描述角度是一位数学专业的学生, 因此我可能会采用一些更易于理解, 但不严格符合课程结构的叙述方式和顺序. 这些特性决定了这份讲义不会有太广泛的适用性.\par
笔者曾经试图撰写过常微分方程、微积分、线性代数等课程的讲义, 但由于时间和精力的限制, 这些讲义都没有完成. 这份讲义是笔者在2025年春季学期和暑期复习这门课程时完成的, 从某种程度上来讲这既是对此前未完成的讲义的一种补偿, 也是对自己本科二年级学习生活的一份总结. 我希望在这份讲义里更多地去体现我对Lebesgue测度的理解和我对分析学的认知. 尽管这些认知可能都是浅显的, 但我仍希望这些想法能够落到具体实际, 以作纪念和方便回顾.\par
除了上述这些想法之外, 我还希望基于此回忆一些学习时的一些有趣的理解, 作为一名可能对分析方向不太感兴趣的学生, 我的这份讲义可能不会带来任何有益的帮助, 反倒可能对基础分析概念的理解产生许多误解, 因此我更希望读者将这份讲义看作漫谈, 而非一份严谨的参考讲义, 同时我也很期待任何同学能够帮助我修正其中的任何错误.\par
\begin{flushright}
\text{Stone Sun}\\
\text{\today}
\end{flushright}

\mainmatter

\btocgroup
\UseTocStyle{chapter}{emph}{toc}
\chapter{集合与映射}
\etocgroup

\section{集合的运算}

\begin{definition}
  设$A,B$是集合, 则有如下集合间运算和关系的定义:\par
  \begin{itemize}
    \item
      $A\bigcup B=\{x|x\in A\text{或}x\in B\}$, 称为$A$和$B$的并集.
    \item
      $A\bigcap B=\{x|x\in A\text{且}x\in B\}$, 称为$A$和$B$的交集.
    \item
      $A\backslash B=\{x|x\in A\text{且}x\notin B\}$, 称为$A$和$B$的差集.
    \item
      $A\subseteq B$表示$A$是$B$的子集, 即$\forall x\in A, x\in B$.
    \item
      $A=B$表示$A$和$B$相等, 即$\forall x, x\in A\Leftrightarrow x\in B$.
    \item 
      若$A\subset S, B=S\backslash A$, 则称$A$是$B$的补集, 记作$A^{c}$.
  \end{itemize}

\end{definition}
针对抽象的集合, 我们有如下定义:\par
\begin{definition}
  设$\varLambda $是一集合, 则称$\{A_{\lambda}\}$是一集族, 其中$\lambda\in\varLambda$.\par
  特别的, 若$\varLambda=\{1,2,\cdots,n\}$, 则称$\{A_{1},A_{2},\cdots,A_{n}\}$是一集列.\par
\end{definition}

集合的运算满足下面的定律:\par
\begin{theorem}
  设$A,B,C$是集合, 则下列命题成立:\par
  \begin{itemize}
    \item $A\cup (B\cup C)=(A\cup B)\cup C$.
    \item $A\cap (B\cap C)=(A\cap B)\cap C$.
    \item $A\bigcap (\bigcup_{\lambda \in \varLambda}B_{\lambda})=\bigcup_{\lambda \in \varLambda}(A\bigcap B_{\lambda})$.
    \item $A\bigcup (\bigcap_{\lambda \in \varLambda}B_{\lambda})=\bigcap_{\lambda \in \varLambda}(A\bigcup B_{\lambda})$.
    \item $(\bigcap_{\lambda \in \varLambda}A_{\lambda})^{c}=\bigcup_{\lambda \in \varLambda}A_{\lambda}^{c}$.
    \item $(\bigcup_{\lambda \in \varLambda}A_{\lambda})^{c}=\bigcap_{\lambda \in \varLambda}A_{\lambda}^{c}$.
  \end{itemize}
\end{theorem}

\begin{proof}
  证明略.
\end{proof}

基于上面给出的集合间运算的性质, 我们作下面的特殊定义, 这些定义将会在未来某些测度论的定理中用到.\par
\begin{definition}
  设$\{A_n\}$是一集列, 则称集合$\underset{N=1}{\overset{\infty}{\bigcap}}\underset{n=N}{\overset{\infty}{\bigcup}}A_n$为集列$\{A_n\}$的上限集, 记作$\limsup_{n \to \infty}A_n$.\par
  设$\{A_n\}$是一集列, 则称集合$\underset{N=1}{\overset{\infty}{\bigcup}}\underset{n=N}{\overset{\infty}{\bigcap}}A_n$为集列$\{A_n\}$的下限集, 记作$\liminf_{n \to \infty}A_n$.\par
\end{definition}

针对上限集和下限集, 我们有如下等价定义:\par
\begin{theorem}
  设$\{A_n\}$是一集列, 则有如下等价定义:\par
  \begin{itemize}
    \item $x\in \limsup_{n \to \infty}A_n\Leftrightarrow \forall N, \exists n_0\geqslant N, x\in A_{n_0}$.
    \item $x\in \liminf_{n \to \infty}A_n\Leftrightarrow \exists N, \forall n\geqslant N, x\in A_{n}$.
  \end{itemize}
\end{theorem}
\begin{proof}
  上面的等价定义是利用命题的存在性和任意性得到的, 这种证明方法会在后续经常用到. 通俗来讲, 即: 并集表示存在性, 交集表示任意性.\par
\end{proof}

现在我们给出下面一些性质, 很好的描述了上限集和下限集在给定的集列上的关系:\par
\begin{theorem}
  给定$\{A_n\}$是一集列, 则下列命题成立:\par
  \begin{itemize}
    \item $(\limsup_{n \to \infty}A_n)^{c}=\liminf_{n \to \infty}(A_n)^{c}$.
    \item $(\liminf_{n \to \infty}A_n)^{c}=\limsup_{n \to \infty}(A_n)^{c}$.
    \item $\bigcap_{n}^{\infty}A_n \subset \liminf_{n \to \infty}A_n\subset \liminf_{n \to \infty}(A_n)\subset \bigcup_{n}^{\infty}A_n$.
  \end{itemize}
\end{theorem}

\begin{proof}
  前两个命题是利用上面给出的等价定义得到的. 这两个命题实际上解释了上限集和下限集的关系.\par
  对于第三个命题, 首先考虑$\liminf_{n \to \infty}A_n\subset \liminf_{n \to \infty}(A_n)\subset$: 若$x\in \liminf_{n \to \infty}A_n$, 则$\exists N, \forall n\geqslant N, x\in A_n$, 这说明$x$在无穷多个$A_n$中出现, 因此$x\in \liminf_{n \to \infty}(A_n)$.\par
  再考虑$\bigcap_{n}^{\infty}A_n \subset \liminf_{n \to \infty}A_n$: 若$x\in \bigcap_{n}^{\infty}A_n$, 则$\forall n, x\in A_n$, 因此$x\in \liminf_{n \to \infty}A_n$.\par
  类似地我们也有$\liminf_{n \to \infty}(A_n)\subset \bigcup_{n}^{\infty}A_n$: 若$x\in \liminf_{n \to \infty}(A_n)$, 则$\exists N, \forall n\geqslant N, x\in A_n$, 因此$x\in \bigcup_{n}^{\infty}A_n$.\par
\end{proof}

\begin{definition}
  集列$\{A_n\}$是收敛的, 当且仅当$\limsup_{n \to \infty}A_n=\liminf_{n \to \infty}A_n$. 此时记$\lim_{n \to \infty} A_n=\limsup_{n \to \infty}A_n=\liminf_{n \to \infty}A_n$为集列$\{A_n\}$的极限.\par
\end{definition}

\begin{corollary}
  若递增集列$\{A_n\}$是收敛的, 则$\limsup_{n \to \infty}A_n=\liminf_{n \to \infty}A_n=\bigcup_{n}^{\infty}A_n$.\par
  若递减集列$\{A_n\}$是收敛的, 则$\limsup_{n \to \infty}A_n=\liminf_{n \to \infty}A_n=\bigcap_{n}^{\infty}A_n$.\par
\end{corollary}
\begin{proof}
  递增集列有$\forall n, A_n\subset A_{n+1}$, 因此$\bigcup_{n}^{\infty}A_n=\lim_{n \to \infty}A_n =\limsup_{n \to \infty}A_n=\liminf_{n \to \infty}A_n$.\par
  递减集列有$\forall n, A_n\supset A_{n+1}$, 因此$\bigcap_{n}^{\infty}A_n=\lim_{n \to \infty}A_n =\limsup_{n \to \infty}A_n=\liminf_{n \to \infty}A_n$.\par
\end{proof}


\btocgroup
\UseTocStyle{chapter}{emph}{toc}
\chapter{Lebesgue测度}
\etocgroup

\btocgroup
\UseTocStyle{chapter}{emph}{toc}
\chapter{可测函数}
\etocgroup

\section{可测函数的定义和性质}

\newpage

\section{可测函数的收敛性}

在讨论函数性质时, 我们自然想到普通微积分中实函数的收敛, 因此我们现在对函数的收敛作如下定义:\par

\begin{definition}
  设$f_n,f$是定义在$E$上的函数, 
  \begin{itemize}
    \item 称$f_n$逐点收敛到$f$, 当且仅当$\forall x\in E, \lim_{n \to \infty}f_n(x)=f(x)$.\par
    \item 称$f_n$一致收敛到$f$, 当且仅当$\forall \varepsilon>0, \exists N\in \mathbb{N}, \forall n\geqslant N, |f_n(x)-f(x)|<\varepsilon$.\par
    \item 称$f_n$几乎处处收敛到$f$, 当且仅当$\exists E_0\subset E, m(E_0) = 0, \forall x\in E\backslash E_0, \lim_{n \to \infty}f_n(x)=f(x)$.\par
    \item 称$f_n$近乎一致收敛到$f$, 当且仅当$\forall \varepsilon>0, \forall \delta > 0, \exists E_0\subset E, m(E_0) < \delta, \exists N\in \mathbb{N}, \forall n\geqslant N, \forall x\in E\backslash E_0, |f_n(x)-f(x)|<\varepsilon$.\par
  \end{itemize}
\end{definition}

\begin{example}
  若$f_n(x)=x^n$, $f(x)=0$, 则$f_n(x)$近乎一致收敛到$f$.\par
  这是由于$\forall \delta>0$, 取$E_0 =\left[1-\frac{\delta}{2},1\right]$, 则$m(E_0)=\frac{\delta}{2}<\delta$, $\forall x\in E\backslash E_0$, $|f_n(x)-f(x)|=|x^n|<\varepsilon$.\par
  此时只需要取$N=\left|\frac{\varepsilon}{\log \left(1-\frac{\delta}{2}\right)}\right|$, 使得$\forall n\geqslant N$, $|x^n|<\varepsilon$.\par
\end{example}

\begin{remark}
  这里$N$是和$\delta$有关的, 但不能和$x$有关, 否则就不再是一致收敛了.\par
\end{remark}

\begin{theorem}
  下列命题等价:\par
  \begin{itemize}
    \item $\exists E_0\subset E, m(E_0) = 0, \forall x\in E\backslash E_0, \lim_{n \to \infty}f_n(x)=f(x)$.
    \item $\lim_{n\to \infty}m(E[f_n \nrightarrow f])=0$.
    \item $\forall x\in E[f_n \nrightarrow f], f_n(x)\nrightarrow f(x)$
  \end{itemize}
\end{theorem}

\begin{lemma}
  $E[f_n \nrightarrow f] = \bigcup_{\varepsilon} \bigcap_{N=1}^{+\infty} \bigcup_{n=N}^{+\infty}E\left[\left|f_n-f\right|\geqslant \varepsilon\right] = \bigcup_{k=1}^{+\infty} \bigcap_{N=1}^{+\infty} \bigcup_{n=N}^{+\infty}E\left[\left|f_n-f\right|\geqslant \frac{1}{k}\right]$.
\end{lemma}

\begin{proof}
  考虑$E[f_n \nrightarrow f]$的定义:\par
  $\exists \varepsilon_0>0, \forall N\in \mathbb{N}, \exists n_0\geqslant N, \left|f_n(x)-f(x)\right|\geqslant \varepsilon_0$.\par
  这说明$\exists \varepsilon_0>0, \forall N\in \mathbb{N}, \exists n_0\geqslant N, x\in E\left[\left|f_n-f\right|\geqslant \varepsilon_0\right]$.\par
  于是有$x\in \bigcup_{\varepsilon} \bigcap_{N=1}^{+\infty} \bigcup_{n=N}^{+\infty}E\left[\left|f_n-f\right|\geqslant \varepsilon_0\right]$.\par
  更进一步考虑$\varepsilon$的任意性, 我们有$x \in \bigcup_{k=1}^{+\infty} \bigcap_{N=1}^{+\infty} \bigcup_{n=N}^{+\infty}E\left[\left|f_n-f\right|\geqslant \frac{1}{k}\right]$.\par
\end{proof}

\begin{remark}
  上面的这种方法在后续的Lebesgue积分中也会用到, 这种方法实际上是利用了$\varepsilon$的任意性来构造一个新的集合, 使得这个集合的测度为0. 但同时考虑到$\varepsilon$是任意的, 因此我们选择$\frac{1}{k}$代替依旧是成立的.\par
\end{remark}

\begin{lemma}
  (Borel-Cantelli) 设$\{A_n\}$是一集列, 则$\sum_{n=1}^{\infty}m(A_n)<\infty \Rightarrow m(\limsup_{n \to \infty}A_n)=0$.
\end{lemma}

\begin{proof}
  首先考虑$e_k$和$E$均可测, 则有下面的等价关系:\par
  $E\backslash \left(\limsup_{k \to \infty}e_k \right)= \liminf_{k\to \infty}\left(E\backslash e_k\right)$.\par
  事实上我们有$\left(\limsup_{k\to \infty}e_k\right)^{c}=\left(\bigcap_{N=1}^{\infty}\left(\bigcup_{k=N}^{\infty}e_k\right)\right)^{c}=\bigcup_{N=1}^{\infty}\left(\bigcup_{k=N}^{\infty}e_k\right)^{c} = \bigcup_{N=1}^{\infty}\bigcap_{k=N}^{\infty}e_k^c$.\par
  下面再考虑$m(E_k)<\frac{1}{2^k}, \forall k$, 则有$m\left(\bigcup_{k=N}^{\infty}E_k\right)\leqslant \frac{1}{2^{N-1}}$. 记$F_N = \bigcup_{k=N}^{\infty}E_k$, 则$m\left(\bigcap_{N=1}^{\infty}F_N\right)\leqslant m\left(F_N\right)\to 0$.\par
\end{proof}

下面我们讨论近乎一致收敛, 几乎处处收敛的关系:\par

\begin{theorem}
  若$f_n,f$在$E$上可测, $f_n$近乎一致收敛到$f$, 则$f_n$几乎处处收敛到$f$.\par
\end{theorem}

\begin{proof}
  $f_n$近乎一致收敛到$f$, 则有$\forall k >0, \exists {e_k}, m(e_k)<\frac{1}{k}, f_n \Rightarrow f \text{on} E\backslash e_k$, 其中$e_k$可测.\par
  往证$m\left(\bigcup_{k=1}^{\infty}\bigcap_{N=1}^{\infty}\bigcup_{n=N}^{\infty} E\left[\left|f_n-f\right|\geqslant \frac{1}{k}\right]\right)=0$. 令$E_0=\limsup_{k \to \infty}e_k$, 则$E\left[f_n\nrightarrow f\right]\subset E_0$.\par
  由Borel-Cantelli引理, 我们有$\sum_{n=1}^{\infty}m\left(E\left[\left|f_n-f\right|\geqslant \frac{1}{k}\right]\right)<\infty$. \par
  因此$m\left(\limsup_{n \to \infty}E\left[\left|f_n-f\right|\geqslant \frac{1}{k}\right]\right)=0$.\par
\end{proof}

\begin{theorem}
  (Egoroff) 设$m(E)<\infty$, 若$f_n$在$E$上几乎处处收敛到$f$, 则$f_n$近乎一致收敛到$f$.\par
\end{theorem}

\begin{proof}
  由$f_n \rightarrow f \text{ a.e.} on E$知, $m\left(\bigcup_{k=1}^{\infty}\bigcap_{N=1}^{\infty}\bigcup_{n=N}^{\infty}E\left[|f_n-f|\geqslant \frac{1}{k}\right]\right)=0$.\par
  记$F_k=\bigcap_{N=1}^{\infty}\bigcup_{n=N}^{\infty}E\left[|f_n-f|\geqslant \frac{1}{k}\right]$, 则$m(F_k)=0, \forall k >0$. 由$m(E)<\infty$有$\lim_{k\to \infty}m(F_k) = m(\lim_{k\to \infty}F_k)=0$.\par
  即$\forall k > 0, \forall \delta >0, \exists k_{\delta}>0, \forall k>k_{\delta}, m(F_{k_{\delta}})<\frac{\delta}{2^{k_{\delta}}}$, 于是$m\left(\bigcup_{k\geqslant k_{\delta}}F_k\right)<\sum_{N=1}^{\infty}\frac{\delta}{2^{k_{\delta}}}=\delta$.\par
  令$e=\bigcup_{n\geqslant k_{\delta}}F_k$, 则$m(e)<\delta$, $\forall x\in E\backslash e, |f_n(x)-f(x)|<\frac{1}{k}$, 这说明$f_n$近乎一致收敛到$f$.\par 
\end{proof}

根据上面的讨论, 我们知道近乎一致收敛有下面的表述方法:
$$\forall \delta >0, \forall \varepsilon >0, \exists E_0\subset E, m(E_0)<\delta, \exists N>0, \forall n\geqslant N, |f_n(x)-f(x)|<\varepsilon, \forall x\in E\backslash E_0.$$

事实上, 我们知道这表示$|f_n(x)-f(x)|\geqslant \varepsilon$的集合的测度应当很小, 而又考虑到$\delta$是任意取的, $N$只与$\delta$有关, 于是我们可以作下面的定义:
$$\forall \delta >0, \forall \varepsilon >0, \exists N, n\geqslant N, m\left(E\left[|f_n-f|\geqslant \varepsilon\right]\right)<\delta.$$

更进一步, 这可以写成
$\forall \varepsilon >0, \lim_{n \to \infty}m\left(E\left[|f_n-f|\geqslant \varepsilon\right]\right) = 0.$

这就有了下面依测度收敛的定义:\par
\begin{definition}
  设$f_n,f$是定义在$E$可测且几乎处处有限的函数, 则$f_n$依测度收敛到$f$, 当且仅当$\forall \varepsilon >0, \lim_{n \to \infty}m\left(E\left[|f_n-f|\geqslant \varepsilon\right]\right) = 0$. 记作$f_n\Rightarrow f$.
\end{definition}

对于这种收敛, 我们继续讨论和其他收敛的关系, 于是有下面的定理:\par

\begin{theorem}
  (Lebesgue) 设$m(E)<\infty$, 若$f_n$几乎处处收敛到$f$, 则$f_n$依测度收敛到$f$.\par
\end{theorem}

\begin{proof}
  由$f_n$几乎处处收敛到$f$, 则$\exists E_0\subset E, m(E_0)=0, \forall x\in E\backslash E_0, \lim_{n \to \infty}f_n(x)=f(x)$.\par
  于是有$\forall \varepsilon >0, \exists N>0, \forall n\geqslant N, |f_n(x)-f(x)|<\varepsilon$, $\forall x\in E\backslash E_0$.\par
  因此$m\left(E\left[|f_n-f|\geqslant \varepsilon\right]\right)\leqslant m(E_0)=0$, 这说明$f_n$依测度收敛到$f$.\par
\end{proof}

\begin{theorem}
  (Riesz) 设$m(E)<\infty$, 若$f_n$依测度收敛到$f$, 则$\exists f_{n_k}$, $f_{n_k}$几乎处处收敛到$f$.\par
\end{theorem}

\begin{proof}
  对于依测度收敛的函数列$f_n$, 总可以取$k>0$使得$e_k=E\left[|f_{n_k}-f|\geqslant \varepsilon\right]$满足$m(e_k)< \frac{1}{2^k}$. 于是$\sum_{k=1}^{\infty}m(e_k)=1<\infty$.\par
  由Borel-Cantelli引理, 我们有$m\left(\limsup_{k \to \infty}e_k\right)=0$, 令$E_0=\limsup_{k \to \infty}e_k$, 则$\forall x \in E\backslash E_0, \exists k_0, \forall k\geqslant k_0, |f_{n_k}(x)-f(x)|<\varepsilon$.\par
  这说明$f_{n_k}$几乎处处收敛到$f$.\par
\end{proof}

\begin{remark}
  几乎处处收敛和依测度收敛不是等价的条件, 例如下面两个例子:\par
  令$f_n(x)=\chi_{\left(\right.0,n\left.\right]}$, 则$f_n(x)$几乎处处收敛到$f(x)=\chi_{\left(\right.0,+\infty\left.\right)}$, 但$f_n$不依测度收敛到$f$, 这是因为我们可以验证$\lim_{n \to \infty}m\left(E\left[|f_n-f|\geqslant 1\right]\right) = +\infty$.\par
  令$f_n(x)$按下面的方法排列: $f_1(x)=\chi_{\left[\right.0,\frac{1}{2}\left.\right]}$, $f_2(x)=\chi_{\left(\right.\frac{1}{2},1\left.\right]}$, $f_3(x)=\chi_{\left[\right.0,\frac{1}{4}\left.\right]}$, $f_4(x)=\chi_{\left(\right.\frac{1}{4},\frac{1}{2}\left.\right]}$,$\cdots$, 则$f_n(x)$依测度收敛到$f(x)=\chi_{\left[\right.0,1\left.\right]}$, 但$f_n$处处不收敛到$f$.\par
\end{remark}

基于上面的结果, 我们下面讨论依测度收敛的性质:\par

\begin{example}
  若$f_n\Rightarrow f$, $f_n\Rightarrow g$, 则$f=g$几乎处处成立.\par
\end{example}

\begin{example}
  若$f_n\Rightarrow f$, $g_n\Rightarrow g$, $f_n\geqslant g_n$在$E$上几乎处处成立, 则$f\geqslant g$在$E$上几乎处处成立.\par
\end{example}

\begin{example}
  若$f_n\Rightarrow f$, $a\in \mathbb{R}$, 则$af_n\Rightarrow af$.\par  
  若$f_n\Rightarrow f$, $g_n \Rightarrow g$, 则$f_n+g_n\Rightarrow f+g$.\par
\end{example}

\begin{corollary}
  若$f_n\Rightarrow 0$, $g_n \Rightarrow 0$, 则$f_ng_n\Rightarrow 0$.\par
\end{corollary}

\begin{proof}
  往证: $\forall \varepsilon >0, \lim_{n \to \infty}m\left(E\left[|f_ng_n|\geqslant \varepsilon\right]\right) = 0$.\par
  记$E_n=E\left[|g_n|>1\right]$, 则$E_n^c=E\left[|g_n|\leqslant 1\right]$.\par
  于是$E\left[|f_ng_n|\geqslant \varepsilon\right]\subset E_n\left[|f_ng_n|\geqslant \varepsilon\right]\cup E_n^c\left[|f_ng_n|\geqslant \varepsilon\right]\subset E_n\cup E\left[|f_n|\geqslant \varepsilon\right]$.\par
  由$f_n,g_n \Rightarrow 0$, $\exists N_1,N_2$使得$n\geqslant N_1, m(E\left[|f_n|\geqslant \varepsilon\right])<\frac{\delta}{2}$, $n\geqslant N_2, m(E\left[|g_n|\geqslant 1\right])<\frac{\delta}{2}$.\par
  取$N = \max\{N_1,N_2\}$, 则$\forall n\geqslant N$, $m(E\left[|f_ng_n|\geqslant \varepsilon\right])<\frac{\delta}{2}+\frac{\delta}{2}=\delta$.\par
  这说明$\lim_{n \to \infty}m\left(E\left[|f_ng_n|\geqslant \varepsilon\right]\right) = 0$. 即$f_ng_n \Rightarrow 0$.\par
\end{proof}

\begin{corollary}
  若$f_n\Rightarrow f$, $g_n = g$有界可测, 则$f_ng\Rightarrow fg$.\par
\end{corollary}

\begin{proof}
  往证: $\forall \varepsilon >0, \lim_{n \to \infty}m\left(E\left[|f_ng-fg|\geqslant \varepsilon\right]\right) = 0$.\par
  由$f_n\Rightarrow f$, $\exists N, \forall n\geqslant N, m(E\left[|f_n-f|\geqslant \frac{\varepsilon}{M}\right])<\delta$.\par
  注意到$E\left[|f_ng-fg|\geqslant \varepsilon\right]\subset  E\left[|f_n-f|\geqslant \frac{\varepsilon}{M}\right]$,其中$M=\sup |g_n|=\sup |g|$.\par
  于是$m\left(E\left[|f_ng-fg|\geqslant \varepsilon\right]\right)\leqslant m\left(E\left[|f_n-f|\geqslant \frac{\varepsilon}{M}\right]\right)<\delta$. 即$f_ng\Rightarrow fg$.\par
\end{proof}

\begin{remark}
  上面的若干例子说明了依测度收敛的函数列在数乘和加法下是封闭的, 但同时注意到第二个推论中的有界性是必要的, 例如下面这个例子:\par
  令$f_n(x)=\chi_{\left(\right.0,n\left.\right]} \frac{1}{x}$, $g(x)=x$, 则$f_n\Rightarrow 0$, 但$f_n\cdot g = \chi_{\left(\right.0,n\left.\right]}$, $f\cdot g = 1$. 此时并没有$\chi_{\left(\right.0,n\left.\right]}\Rightarrow \chi_{\left(0,\infty\right)}$.\par
\end{remark}

\newpage

\section{可测函数的连续性}

上面讨论的许多收敛都是在可测函数的基础上进行的, 现在我们讨论可测函数的连续性, 这一点是必须的, 这是因为后续我们在建立可积函数, 可测函数与联系函数的关系时会更进一步讨论连续的特点. \par

\begin{lemma}
  若$E$是一可测集, 则$C(E)\subset \mathcal{M} (E)$.
\end{lemma}

\begin{proof}
  由$E$是可测集, 因此$E[f>a]=E\cap f^{-1}\left(a,\infty\right), \forall a \in \mathbb{R}$. \par
  由连续映射的定义知, $f^{-1}\left(a,\infty\right)$是开集. 于是$E\cap f^{-1}\left(a,\infty\right)$可测.\par
  即$E[f>a],\forall a \in \mathbb{R}$是可测的. 于是$f$是可测函数.\par
\end{proof}

上面这个定理说明了连续函数在可测集上是可测的, 下面我们为了得到更强结果的连续性条件, 我们首先考虑下面的一系列问题.\par

\begin{lemma}
  设$A,B$均为闭集, $A\cap B=\Phi$, 则$\exists g\in C(\mathbb{R}^n), \left.g\right|_A=1, \left.g\right|_B=0$.且$g$满足$|g(x)|\leqslant 1,\forall x\in \mathbb{R}^n$. \par
\end{lemma}

\begin{proof}
  可以自然想到距离定义, 于是有$g=\frac{d(x,B)}{d(x,A)+d(x,B)}$.\par
  其中$d(x,A)=\inf_{y\in A}|x-y|$, $d(x,B)=\inf_{y\in B}|x-y|$. 同样可以验证$g(x)$是满足上面的条件的.\par
\end{proof}

于是我们得到下面的定理, 具体说明了连续函数和可测函数的关系:\par

\begin{theorem}
  (Lusin I) 设$E$是一可测集, $f\in \mathcal{M} (E)$, 则$\forall \delta >0, \exists F\subset E, m(E\backslash F)<\delta,$ 其中$F$是闭集, 且$f$在$F$上连续.\par
\end{theorem}

\begin{proof}
  由$f\in \mathcal{M} (E)$有, $\exists \psi_k \rightarrow f$, 其中$\psi_k$是简单函数.\par
  若$f$有界可测, 则$\psi_k$有界, $\exists F_k, m(E\backslash F_k)<\frac{\delta}{2^{k+1}}$, 其中$F_k$是闭集. 此时$\psi_k$在$F_k$上连续.\par
  令$F=\bigcap_{k=1}^{\infty}F_k$, 则$m(E\backslash F)\leqslant \sum_{k=1}^{\infty}m(E\backslash F_k)<\delta$. 于是$f$在$F$上连续.\par
  若$f$无界, 构造$g(x)=\frac{f(x)}{1+|f(x)|}$, 则$g$是有界的, 且$g\in \mathcal{M} (E)$.\par
  于是$\exists F\subset E, m(E\backslash F)<\delta$, 其中$F$是闭集, 且$g$在$F$上连续.\par
  由$g(x)=\frac{f(x)}{1+|f(x)|}$, $g(x)+g(x)|f(x)|=f(x)$, 即$g(x)+|g(x)|f(x)=f(x)$, $f(x)=\frac{g(x)}{1-|g(x)|}$, $f$在$F$上连续.\par
\end{proof}

事实上我们还有$Lusin II$, 可以表示如下:\par

\begin{theorem}
  (Lusin II) 设$E$是一可测集, $f\in \mathcal{M} (E)$, 则$\forall \delta >0, \exists g\in C(\mathbb{R}^n), \exists F\subset E, m(E\backslash F)<\delta,$ 其中$F$是闭集, 且$g(x)=f(x),\forall x \in F$.\par
\end{theorem}

\begin{proof}
  由Lusin I, 我们有$\exists F\subset E, m(E\backslash F)<\delta,$ 其中$F$是闭集, 且$f$在$F$上连续.\par
  令$g=\left.f\right|_{F}$, 则$g\in C(F)$.\par
  为了证明$g\in C(\mathbb{R}^n)$, 我们构造$g_k(x)=\left\{\begin{array}{ll}
    g(x), & x\in F, \\
    \frac{1}{k}, & x\in E\backslash F.
  \end{array}\right.$\par
  则$g_k\in C(\mathbb{R}^n)$, 且$\forall x\in F, g_k(x)=g(x)$, $\forall x\in E\backslash F, g_k(x)=\frac{1}{k}$.\par
  于是$\forall x\in E, g_k(x)\rightarrow g(x)$, 这说明$g_k\rightarrow g$在$E$上收敛.\par
  由于$g_k$是连续的, 因此$g\in C(\mathbb{R}^n)$.\par
\end{proof}
\newpage

\chapter{Lebesgue积分}

\section{非负可测函数的积分}

在讨论一般的积分之前, 我们先定义下方图形, 这个概念实际上是Lebesgue积分的几何意义.\par

\begin{definition}
  设$f$是定义在$E$上的非负可测函数, 则称$f$的下方图形称为$G(E,f)=\{(x,z)\in \mathbb{R}^2: x\in E, 0\leqslant z< f(x)\}$.\par
\end{definition}

\begin{remark}
  这里我们要求$0\leqslant z< f(x)$, 实际上$0\leqslant z\leqslant f(x)$所表达的几何意义是一样的, 但按照后者定义证明函数Lebesgue可测时需要更深层次的分析技术.\par
\end{remark}

下面我们依次定义特征函数和简单函数的Lebesgue积分, 后面我们将用这些函数来定义一般的非负可测函数的Lebesgue积分.\par

\begin{definition}
  设$f(x)=\chi_{A}(x)$, 则$\int_{E}f(x)\mathrm{d}x=1\cdot m(A\cap E)$称为$f(x)$在$E$上的Lebesgue积分.\par
  设$f(x)=\sum_{k=1}^{N}c_k\cdot \chi_{A_k}(x)$, 则$\int_{E}f(x)\mathrm{d}x=\sum_{k=1}^{N}c_k\cdot m(A_k\cap E)$称为$f(x)$在$E$上的Lebesgue积分.\par
\end{definition}

\begin{remark}
  显然可以得到特征函数和简单函数的Lebesgue积分的几何意义就是对应的下方图形的测度.\par
\end{remark}

\begin{lemma}
  设$f$和$g$是定义在$E$上的非负简单函数, 则:\par
  \begin{itemize}
    \item $\int_{E}f(x)\mathrm{d}x\geqslant 0$.
    \item 若$f(x)\leqslant g(x)$, 则$\int_{E}f(x)\mathrm{d}x\leqslant \int_{E}g(x)\mathrm{d}x$.
    \item $\int_{E}(af(x)+bg(x))\mathrm{d}x=a\int_{E}f(x)\mathrm{d}x+b\int_{E}g(x)\mathrm{d}x$, 其中$a,b\in \mathrm{R}^{+}$.
  \end{itemize}
\end{lemma}

下面我们定义一般的非负可测函数的Lebesgue积分.\par

\begin{definition}
  设$f:E\rightarrow [c,d]$是非负可测函数, 则$f$在$E$上有Riemann积分$\lim_{|\lambda|\rightarrow 0}\sum_{k=1}^{N}f(\xi_k)\cdot \Delta x_k$, 类似地, 我们有Lebesgue积分$\lim_{|\lambda|\rightarrow 0}\sum_{k=1}^{N}\eta_k\cdot m(E_k)$.\par
\end{definition}

不难注意到, 这里的$\eta_k$是$f$在$E_k$上的平均值, 而$m(E_k)$是$E_k$的测度, 因此$\eta_k\cdot m(E_k)$实际上是下方图形的面积. 但从严格证明的角度, 这种方式并不方便处理抽象测度的集合, 因此我们考虑下面这种定义方式:\par

\begin{definition}
  设$f:E\rightarrow [c,d]$是非负可测函数, 则$f$在$E$上有Lebesgue积分$\lim_{k\to \infty}\int_{E}\phi_k(x)\mathrm{d}x$, 其中$\phi_k(x)$为单增简单函数, 且$\phi_k(x)\rightarrow  f(x)$, $\forall x\in E$.\par
\end{definition}

这种定义方式看似简单, 但实际上需要证明$\phi_k(x)$的存在性, 并且若出现$\phi_k(x)$和$\psi_k(x)$时, 需要进一步说明$\lim_{k\to \infty}\int_{E}\phi_k(x)\mathrm{d}x=\lim_{k\to \infty}\int_{E}\psi _k(x)\mathrm{d}x$.\par

\begin{definition}
  设$f:E\rightarrow [c,d]$是非负可测函数, 则$f$在$E$上有Lebesgue积分$\sup\left\{\int_{E}\phi(x)\mathrm{d}x: 0\leqslant \phi(x) \leqslant f(x)\right\}$, 其中 $g(x)$是一列非负简单函数.\par
\end{definition}

后面的各种Lebesgue积分的问题中我们都只采用第三种方式定义, 但可以证明三种定义方式时等价的.\par

\begin{lemma}
  设$f$和$g$是定义在$E$上的非负可测函数, 则:\par
  \begin{itemize}
    \item $\int_{E}f(x)\mathrm{d}x\geqslant 0$.
    \item 若$f(x)\leqslant g(x)$, 则$\int_{E}f(x)\mathrm{d}x\leqslant \int_{E}g(x)\mathrm{d}x$.
    \item $\int_{E}(af(x)+bg(x))\mathrm{d}x=a\int_{E}f(x)\mathrm{d}x+b\int_{E}g(x)\mathrm{d}x$, 其中$a,b\in \mathrm{R}^{+}$.
  \end{itemize}
\end{lemma}

\begin{theorem}
  设$f$是定义在$E$上的非负可测函数, 则$\int_{E}f(x)\mathrm{d}x=m(G(E,f))$.\par
\end{theorem}

\begin{proof}
  由$f$是非负可测函数, 则$\exists \phi_k(x)$是单增简单函数, 且$\phi_k(x)\rightarrow f(x)$, $\forall x\in E$.\par
  于是$m(G(E,\phi_k))\leqslant m(G(E,f))$. 于是$\int_{E}f(x)\mathrm{d}x\leqslant m(G(E,f))$. 下证$\int_{E}f(x)\mathrm{d}x\geqslant m(G(E,f))$:\par
  $\forall (x,z)\in G(E,f), x \in E, 0\leqslant z<f(x)$, 则$\exists \phi_{k0}(x)$满足$0\leqslant z<\phi_k(x)<f(x)$, $\forall x\in E$.\par
  即$(x,z)\in G(E,\phi_{k_0})\subset \bigcup_{k=1}^{\infty}G(E,\phi_{k})\subset G(E,f)$.\par
  由$(x,z)$的任意性, 有$m(G(E,f))\subset \bigcup_{k=1}^{\infty}(G(E,\phi_k))$.\par
  于是$m(G(E,f))=\lim_{k\to \infty}m(G(E,\phi_k))$. 即$\int_{E}f(x)\mathrm{d}x\geqslant m(G(E,f))$.\par
\end{proof}

下面围绕非负可测函数的Lebesgue积分, 我们讨论一些性质.\par

\begin{corollary}
  设$f\in \mathcal{M} (E)$, $f\geqslant 0$, 若$\int_{E}f(x)\mathrm{d}x=0$, 则$f(x)=0$, a.e. on $E$.\par
\end{corollary}

\begin{proof}
  假设$m(E[f\ne 0])>0$, 则$m(E[f> 0])>0$, 于是$m(\bigcup_{n=1}^{\infty}E[f>\frac{1}{n}])$.\par
  于是$\exists n_0>0, \delta>0, m(E[f>\frac{1}{n_0}])\geqslant \delta>0$.\par
  此时就会有$\int_{E}f(x)\mathrm{d}x\geqslant \frac{1}{n_0}\cdot m(E[f>\frac{1}{n_0}])\geqslant \frac{\delta}{n_0}>0$.\par 
\end{proof}

\begin{corollary}
  设$f\in \mathcal{M} (E)$, $m(E)\geqslant 0$, $f>0$ a.e. on $E$, 则$\int_{E}f(x)\mathrm{d}x>0$.\par
\end{corollary}

\begin{proof}
  由$f>0$ a.e. on $E$, 则$m(E[f>0])>0$. 于是$\forall \varepsilon >0, m(E[f>\varepsilon])>0$.\par
  于是$\int_{E}f(x)\mathrm{d}x\geqslant \varepsilon\cdot m(E[f>\varepsilon])>0$.\par
\end{proof}


\newpage

\section{一般可测函数的积分}


\backmatter


\end{document}
