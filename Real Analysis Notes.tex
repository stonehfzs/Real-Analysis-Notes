% \documentclass[lang=cn,newtx,10pt,scheme=chinese]{elegantbook}
\documentclass[theorem=false,mathfont=none,openany,sub3section]{easybook}

\usepackage[lang=cn]{eb-elegantbook}
\usepackage{lmodern}
\usepackage{codehigh}
\lstset{moreemph={nofont,thmenv}}

\renewcommand{\rmdefault}{lmr}
\renewcommand{\sfdefault}{lmss}
\renewcommand{\ttdefault}{lmtt}

\title{Real Analysis}
\subtitle{Lecture Notes}
\author{Stone Sun}
%\institute{Ocean University of China}
\date{\today}
%\version{}
\bioinfo{联系方式}{hefengzhishui@outlook.com}

%\extrainfo{注意:Elegant系列模板自 2023 年 1 月 1 日开始,不再更新和维护!}

\setcounter{tocdepth}{3}

\logo{logo.jpg}
\cover{cover.jpg}

% 本文档命令
% \usepackage{array}
\newcommand{\ccr}[1]{\makecell{{\color{#1}\rule{1cm}{1cm}}}}

% 修改标题页的橙色带
\definecolor{customcolor}{RGB}{32,178,170}
\colorlet{coverlinecolor}{customcolor}
% \usepackage{cprotect}

\addbibresource[location=local]{reference.bib} % 参考文献,不要删除

% 定义\btocgroup和\etocgroup命令把目录风格限制在组内,使其局部生效
\newcommand{\btocgroup}[1][toc]{\addtocontents{#1}{\string\begingroup}}
\newcommand{\etocgroup}[1][toc]{\addtocontents{#1}{\string\endgroup}}
%\SetTocStyle{chapter}{emph}{tocformat+ = \color{black}}
\let\ls\lstinline
\ebhdrset{footnotetype=flush}
\ctexset{
  paragraph/numbering=false,
  paragraph/beforeskip=1explus.2ex
  }
\SetTocStyle{subsubsection}{sub2}{
  tocindent=3.8em,
  tocformat+=\color{blue}
  }
\UseTocStyle{subsubsection}{sub2}{toc}

\begin{document}

\maketitle
\begin{center}
谨以此篇, 献给热爱分析的你.\par
\end{center}
\frontmatter

\begingroup
\renewcommand{\familydefault}{\rmdefault}
\tableofcontents
\endgroup

\newpage
\begin{center}
\Large
\textbf{前言}\par
\end{center}

\hspace{2em}

这是一份关于实分析(又名实变函数)的讲义, 主要涵盖了$\mathbb{R}^n$上的集合论与测度论, 同时讨论了Lebesgue可测、Lebesgue积分的基本概念和定理. 这份讲义是基于中国海洋大学的实变函数课程的讲义和笔记而写成的, 也参考了其他一些教材和讲义.\par
值得注意的是, 这门课程在不同开课学院的所占学分和所需学时是不同的. 所以这份讲义可能更适合每周3学时的同学学习参考.\par
这份讲义的描述角度是一位数学专业的学生, 因此我可能会采用一些更易于理解, 但不严格符合课程结构的叙述方式和顺序. 这些特性决定了这份讲义不会有太广泛的适用性.\par
笔者曾经试图撰写过常微分方程、微积分、线性代数等课程的讲义, 但由于时间和精力的限制, 这些讲义都没有完成. 这份讲义是笔者在2025年春季学期和暑期复习这门课程时完成的, 从某种程度上来讲这既是对此前未完成的讲义的一种补偿, 也是对自己本科二年级学习生活的一份总结. 我希望在这份讲义里更多地去体现我对Lebesgue测度的理解和我对分析学的认知. 尽管这些认知可能都是浅显的, 但我仍希望这些想法能够落到具体实际, 以作纪念和方便回顾.\par
除了上述这些想法之外, 我还希望基于此回忆一些学习时的一些有趣的理解, 作为一名可能对分析方向不太感兴趣的学生, 我的这份讲义可能不会带来任何有益的帮助, 反倒可能对基础分析概念的理解产生许多误解, 因此我更希望读者将这份讲义看作漫谈, 而非一份严谨的参考讲义, 同时我也很期待任何同学能够帮助我修正其中的任何错误.\par
\begin{flushright}
\text{Stone Sun}\\
\text{\today}
\end{flushright}

\mainmatter

\btocgroup
\UseTocStyle{chapter}{emph}{toc}
\chapter{基础集合论}
\etocgroup

\section{集合的运算}

\begin{definition}
  设$A,B$是集合, 则有如下集合间运算和关系的定义:\par
  \begin{itemize}
    \item
      $A\bigcup B=\{x|x\in A\text{或}x\in B\}$, 称为$A$和$B$的并集.
    \item
      $A\bigcap B=\{x|x\in A\text{且}x\in B\}$, 称为$A$和$B$的交集.
    \item
      $A\backslash B=\{x|x\in A\text{且}x\notin B\}$, 称为$A$和$B$的差集.
    \item
      $A\subseteq B$表示$A$是$B$的子集, 即$\forall x\in A, x\in B$.
    \item
      $A=B$表示$A$和$B$相等, 即$\forall x, x\in A\Leftrightarrow x\in B$.
    \item 
      若$A\subset S, B=S\backslash A$, 则称$A$是$B$的补集, 记作$A^{c}$.
  \end{itemize}

\end{definition}
针对抽象的集合, 我们有如下定义:\par
\begin{definition}
  设$\varLambda $是一集合, 则称$\{A_{\lambda}\}$是一集族, 其中$\lambda\in\varLambda$.\par
  特别的, 若$\varLambda=\{1,2,\cdots,n\}$, 则称$\{A_{1},A_{2},\cdots,A_{n}\}$是一集列.\par
\end{definition}

集合的运算满足下面的定律:\par
\begin{theorem}
  设$A,B,C$是集合, 则下列命题成立:\par
  \begin{itemize}
    \item $A\cup (B\cup C)=(A\cup B)\cup C$.
    \item $A\cap (B\cap C)=(A\cap B)\cap C$.
    \item $A\bigcap (\bigcup_{\lambda \in \varLambda}B_{\lambda})=\bigcup_{\lambda \in \varLambda}(A\bigcap B_{\lambda})$.
    \item $A\bigcup (\bigcap_{\lambda \in \varLambda}B_{\lambda})=\bigcap_{\lambda \in \varLambda}(A\bigcup B_{\lambda})$.
    \item $(\bigcap_{\lambda \in \varLambda}A_{\lambda})^{c}=\bigcup_{\lambda \in \varLambda}A_{\lambda}^{c}$.
    \item $(\bigcup_{\lambda \in \varLambda}A_{\lambda})^{c}=\bigcap_{\lambda \in \varLambda}A_{\lambda}^{c}$.
  \end{itemize}
\end{theorem}

\begin{proof}
  证明略.
\end{proof}

基于上面给出的集合间运算的性质, 我们作下面的特殊定义, 这些定义将会在未来某些测度论的定理中用到.\par
\begin{definition}
  设$\{A_n\}$是一集列, 则称集合$\underset{N=1}{\overset{\infty}{\bigcap}}\underset{n=N}{\overset{\infty}{\bigcup}}A_n$为集列$\{A_n\}$的上限集, 记作$\limsup_{n \to \infty}A_n$.\par
  设$\{A_n\}$是一集列, 则称集合$\underset{N=1}{\overset{\infty}{\bigcup}}\underset{n=N}{\overset{\infty}{\bigcap}}A_n$为集列$\{A_n\}$的下限集, 记作$\liminf_{n \to \infty}A_n$.\par
\end{definition}

针对上限集和下限集, 我们有如下等价定义:\par
\begin{theorem}
  设$\{A_n\}$是一集列, 则有如下等价定义:\par
  \begin{itemize}
    \item $x\in \limsup_{n \to \infty}A_n\Leftrightarrow \forall N, \exists n_0\geqslant N, x\in A_{n_0}$.
    \item $x\in \liminf_{n \to \infty}A_n\Leftrightarrow \exists N, \forall n\geqslant N, x\in A_{n}$.
  \end{itemize}
\end{theorem}
\begin{proof}
  上面的等价定义是利用命题的存在性和任意性得到的, 这种证明方法会在后续经常用到. 通俗来讲, 即: 并集表示存在性, 交集表示任意性.\par
\end{proof}

现在我们给出下面一些性质, 很好的描述了上限集和下限集在给定的集列上的关系:\par
\begin{theorem}
  给定$\{A_n\}$是一集列, 则下列命题成立:\par
  \begin{itemize}
    \item $(\limsup_{n \to \infty}A_n)^{c}=\liminf_{n \to \infty}(A_n)^{c}$.
    \item $(\liminf_{n \to \infty}A_n)^{c}=\limsup_{n \to \infty}(A_n)^{c}$.
    \item $\bigcap_{n}^{\infty}A_n \subset \liminf_{n \to \infty}A_n\subset \liminf_{n \to \infty}(A_n)\subset \bigcup_{n}^{\infty}A_n$.
  \end{itemize}
\end{theorem}

\begin{proof}
  前两个命题是利用上面给出的等价定义得到的. 这两个命题实际上解释了上限集和下限集的关系.\par
  对于第三个命题, 首先考虑$\liminf_{n \to \infty}A_n\subset \liminf_{n \to \infty}(A_n)\subset$: 若$x\in \liminf_{n \to \infty}A_n$, 则$\exists N, \forall n\geqslant N, x\in A_n$, 这说明$x$在无穷多个$A_n$中出现, 因此$x\in \liminf_{n \to \infty}(A_n)$.\par
  再考虑$\bigcap_{n}^{\infty}A_n \subset \liminf_{n \to \infty}A_n$: 若$x\in \bigcap_{n}^{\infty}A_n$, 则$\forall n, x\in A_n$, 因此$x\in \liminf_{n \to \infty}A_n$.\par
  类似地我们也有$\liminf_{n \to \infty}(A_n)\subset \bigcup_{n}^{\infty}A_n$: 若$x\in \liminf_{n \to \infty}(A_n)$, 则$\exists N, \forall n\geqslant N, x\in A_n$, 因此$x\in \bigcup_{n}^{\infty}A_n$.\par
\end{proof}

\begin{definition}
  集列$\{A_n\}$是收敛的, 当且仅当$\limsup_{n \to \infty}A_n=\liminf_{n \to \infty}A_n$. 此时记$\lim_{n \to \infty} A_n=\limsup_{n \to \infty}A_n=\liminf_{n \to \infty}A_n$为集列$\{A_n\}$的极限.\par
\end{definition}

\begin{corollary}
  若递增集列$\{A_n\}$是收敛的, 则$\limsup_{n \to \infty}A_n=\liminf_{n \to \infty}A_n=\bigcup_{n}^{\infty}A_n$.\par
  若递减集列$\{A_n\}$是收敛的, 则$\limsup_{n \to \infty}A_n=\liminf_{n \to \infty}A_n=\bigcap_{n}^{\infty}A_n$.\par
\end{corollary}
\begin{proof}
  递增集列有$\forall n, A_n\subset A_{n+1}$, 因此$\bigcup_{n}^{\infty}A_n=\lim_{n \to \infty}A_n =\limsup_{n \to \infty}A_n=\liminf_{n \to \infty}A_n$.\par
  递减集列有$\forall n, A_n\supset A_{n+1}$, 因此$\bigcap_{n}^{\infty}A_n=\lim_{n \to \infty}A_n =\limsup_{n \to \infty}A_n=\liminf_{n \to \infty}A_n$.\par
\end{proof}

\newpage

\section{集合的对等}

\newpage

\section{集合的基数}

\newpage

\section{距离空间}

\newpage

\section{开集、闭集及其构造}

\newpage




\btocgroup
\UseTocStyle{chapter}{emph}{toc}
\chapter{Lebesgue测度}

\etocgroup

\section{外测度}

在一般的微积分中, 我们会选择区间和对应的长度来衡量集合的大小, 因为我们发现这是定义和运算微积分的前提. 而对于在第一章中定义的各种不同集合, 我们也需要类似的一种方法来定义集合的大小, 这就自然考虑到下面提及的外测度和测度.\par

\begin{definition}
  设$E\subset \mathbb{R}^n$, 则称$E$的外测度为
  \begin{equation*}
    m^{*}(E)=\inf\left\{\sum_{i=1}^{\infty}|I_i|: E\subseteq \bigcup_{i=1}^{\infty}I_i\right\},
  \end{equation*}
  其中$I_i$是$\mathbb{R}^n$中的开区间, $|I_i|$是$I_i$的体积.\par
\end{definition}

\begin{example}
  考虑两个最特殊的集合$\mathbb{R}$和$\phi$, 我们有$m^{*}(\mathbb{R})=\infty$和$m^{*}(\phi)=0$.\par
\end{example}

\begin{proposition}
  外测度具有如下性质:\par
  \begin{itemize}
    \item (外测度非负性) $m^{*}(E)\geqslant 0$.
    \item (外测度的等价定义) $m^{*}(E)=a<\infty \Leftrightarrow \forall \varepsilon >0, \exists \{I_k\}, E\subset \bigcup_{k=1}^{\infty}I_k, \sum_{k=1}^{\infty}|I_k|<a+\varepsilon$
  \end{itemize}
\end{proposition}

\begin{proof}
  考虑到外测度是由若干开区间的并构成的, 且开区间的体积总是非负的, 因此外测度一定非负.\par
  根据下确界的定义, 如果$m^{*}(E)=\inf\left\{\sum_{i=1}^{\infty}|I_i|: E\subseteq \bigcup_{i=1}^{\infty}I_i\right\}$.\par
  则$\forall \varepsilon >0$, 都存在一组开区间$\{I_k\}$, 使得$E\subset \bigcup_{k=1}^{\infty}I_k$且$\sum_{k=1}^{\infty}|I_k|<m^{*}(E)+\varepsilon$.\par
\end{proof}

\begin{corollary}
  $\mathbb{R}^n$中可列点集的外测度为0.\par
\end{corollary}

\begin{proof}
  首先考虑$\mathbb{R}^1$上的情形, 设$E=\{r_1,r_2,\cdots\}$, 并且首先考虑$r_1$, 此时应当有$\forall \delta >0, I_{\delta} = (p-\delta,p+\delta)\supset E$.\par
  于是$m^{*}(\{r_1\})=\inf\left\{\sum_{i=1}^{\infty}|I_i|: E\subseteq \bigcup_{i=1}^{\infty}I_i\right\}\leqslant |I_{\delta}|=2\delta.$\par
  令$\delta \to 0$, 则$m^{*}(\{r_1\})=0$.\par
  对于$r_2$, 同理有$m^{*}(\{r_2\})=0$.\par
  于是对于$\{r_1,r_2,\cdots\}$, 由于$\{r_1,r_2,\cdots\}$是可列点集, 因此可以找到一组开区间$\{I_k\}$, 使得$E\subset \bigcup_{k=1}^{\infty}I_k$且$\sum_{k=1}^{\infty}|I_k|<\varepsilon$.\par
  而对于$\mathbb{R}^n$上的情形, 可将其看作$\mathbb{R}^1$上的可列点集的乘积, 因此同样有$m^{*}(E)=0$.\par
\end{proof}

\begin{proposition}
  (外测度的单调性) 若$E\subset F$, 则$m^{*}(E)\leqslant m^{*}(F)$.\par
\end{proposition}

\begin{proof}
  由外测度的定义, $\exists \{I_k\}, F\subset \bigcup_{k=1}^{\infty}I_k, \sum_{k=1}^{\infty}|I_k|<m^{*}(F)+\varepsilon$.\par
  因为$E\subset F$, 所以$E\subset \bigcup_{k=1}^{\infty}I_k$, 因此$m^{*}(E)\leqslant m^{*}(F)$.\par
\end{proof}

\begin{corollary}
  $m^{*}((a,b))=b-a$, $m^{*}([a,b])=b-a$.\par
\end{corollary}

\begin{proof}
  显然$(a,b)$和$[a,b]$都可以被开区间$(a-\varepsilon,b+\varepsilon)$所覆盖, 且其体积为$b-a+2\varepsilon$, 因此当$\varepsilon \to 0$时, 有$m^{*}((a,b))=b-a$和$m^{*}([a,b])=b-a$.\par
\end{proof}

\begin{remark}
  上述结论推广至$\mathbb{R}^n$上也是成立的, 综合上面两个推论的证明即可.\par
  对于区间某一端点为开的情形, 结论也是成立的, 这里只需要考虑开区间的另一端即可.\par
\end{remark}

\begin{theorem}
  (外测度的平移不变性) 设$E\subset \mathbb{R}^n$, $x+E=\{x+y:y\in E\}$, 则$m^{*}(E)=m^{*}(x+E)$.\par
\end{theorem}

\begin{proof}
  这是很显然的, 因为对于每个满足$m^{*}(E)=\inf\left\{\sum_{i=1}^{\infty}|I_i|: E\subseteq \bigcup_{i=1}^{\infty}I_i\right\}$的开区间$I_i$, 都可以通过平移得到满足$m^{*}(x+E)=\inf\left\{\sum_{i=1}^{\infty}|J_i|: x+E\subseteq \bigcup_{i=1}^{\infty}J_i\right\}$的开区间.\par
  而对于每个$I_i$和$J_i$, 都有$|I_i|=|J_i|$, 因此有$m^{*}(E)=m^{*}(x+E)$.\par
\end{proof}

上面讨论了外测度的各种基本性质, 在一般微积分中我们可以自然做集合的各种运算, 其长度都是良好定义的, 因此接下来我们将讨论外测度的可加性.\par

\begin{theorem}
  (外测度的次可加性) 设$E_1, \cdots, E_n\subset \mathbb{R}^n$, 则有$m^{*}\left(\bigcup_{k=1}^{\infty}E_k\right)\leqslant \sum_{k=1}^{\infty}m^{*}(E_k)$.\par
\end{theorem}

\begin{proof}
  若$\exists E_{n_0}, m^{*}(E_{n_0})=\infty$, 则$\sum_{k=1}^{\infty}m^{*}(E_k)=\infty$, 此时显然成立.\par
  若$\forall n, m^{*}(E_n)<\infty$, 则$\forall E_k$ 都有$\forall \varepsilon >0, \exists \{I_{k_j}\}, E_k\subset \bigcup_{k=1}^{\infty}I_{k_j}, m^{*}(E_k)\leqslant \sum_{j=1}^{\infty}|I_{k_j}|<m^{*}(E_k)+\frac{\varepsilon}{2^k}$.\par
  于是$\sum_{k=1}^{\infty}m^{*}(E_k)+\varepsilon \geqslant m^{*}(\bigcup_{k=1}^{\infty}E_k)$. 即成立.\par
\end{proof}

\begin{theorem}
  (外测度的分离可加性) 设$E_1, \cdots, E_n\subset \mathbb{R}^n$, 若$\forall i,j, \rho(E_i,E_j)>0$, 则有$m^{*}\left(\bigcup_{k=1}^{\infty}E_k\right)= \sum_{k=1}^{\infty}m^{*}(E_k)$.\par
\end{theorem}

\begin{proof}
  若$\forall i,j, \rho(E_i,E_j)>0$, 则总可以把每个$E_i$都拆成若干不相交的小区间, 即$E_i=\bigcup_{k=1}^{\infty}E_{i_k}$, $E_j=\bigcup_{k=1}^{\infty}E_{j_k}$且$\left(\bigcup_{k=1}^{\infty}E_{i_k}\right)\cap \left(\bigcup_{k=1}^{\infty}E_{j_k}\right) = \phi $.\par
  于是对于每个$E_i$, 都可以找到恰好满足$m^{*}(E_i)=\sum_{i=1}^{\infty}m^{*}(E_{i_k})$的开区间, 使得$E_i\subset \bigcup_{k=1}^{\infty}I_{i_k}$.\par
  因此有$m^{*}\left(\bigcup_{i=1}^{\infty}E_i\right)= \sum_{i=1}^{\infty}m^{*}(E_i)$.\par
\end{proof}

\begin{remark}
  一般情形的集合的外测度不满足可列可加性.\par
\end{remark}

\newpage

\section{可测集及其测度}

上一节我们提到外测度的可加性是必须满足一定条件才能成立的, 因此对于更特殊的集合, 始终满足可加性的集合就说明其存在一种良好的性质. 因此这里我们就从可加性的角度来考虑集合可测性.\par

\begin{theorem}
  设$E$和$F$是定义在$\mathbb{R}^n$上的开区间, 若$E\cap F = \phi$, 则$m^{*}(E\cup F)= m^{*}(E)+m^{*}(F)$.\par
\end{theorem}

\begin{proof}
  首先由次可加性有$m^{*}(E\cup F)\leqslant m^{*}(E)+m^{*}(F)$. 因此下面证明另一个方向的不等式.\par
  考虑$E\subset \bigcup_{i=1}^{\infty}(E\cap I_i)$, 其中$I_i$是开区间, 于是有$m^{*}(E)\leqslant \sum_{i=1}^{\infty}|E\cap I_i|$.\par
  对于$F$, 也有$m^{*}(F)\leqslant \sum_{i=1}^{\infty}|F\cap J_i|$.\par
  由于$E\cap F = \phi$, $I_i=(I_i\cap E)\cup (I_i\cap F)\cup (I_i\backslash (E\cup F))$. 因此$|I_i|\geqslant |I_i\cap E|+|I_i\cap F|$.\par
  由外测度定义可知, $\forall \varepsilon >0, \exists \{I_i\}, m^{*}(E\cup F)\leqslant \sum_{i=1}^{\infty}|I_i|\leqslant m^{*}(E\cup F)+\varepsilon$.\par
  于是$m^{*}(E\cup F)+\varepsilon \geqslant \sum_{i=1}^{\infty}|I_i\cap E|+|I_i\cap F|\geqslant m^{*}(E)+m^{*}(F)$.\par
  因此有$m^{*}(E\cup F)= m^{*}(E)+m^{*}(F)$.\par
\end{proof}

\begin{remark}
  如果考虑集合的等价表示, 上面的定理还有另一种写法.\par
\end{remark}

\begin{corollary}
  设$E$和$F$是定义在$\mathbb{R}^n$上的开区间, 若$E\cap F = \phi$, $E\cup F=I$ 则$m^{*}(E\cup F)= m^{*}(I\cap E)+m^{*}(I\cap E^c)$.\par
\end{corollary}

\begin{corollary}
  设$E$和$F$是定义在$\mathbb{R}^n$上的任意集合, $\exists I$是开区间, 使得$E\subset I$, $F\subset I^c$, 则$m^{*}(E\cup F)= m^{*}(E)+m^{*}(F)$.\par
\end{corollary}

\begin{proof}
  类似地, 作$m^{*}(E)\leqslant \sum_{k=1}^{\infty}|I\cap I_k|$和$m^{*}(F)\leqslant \sum_{k=1}^{\infty}|I^c\cap I_k|$.\par
  则由$E\cap F=\phi$, $I_k=(I_k\cap I)\cup (I_k\cap I^c)$, 此时有$|I_k|=|I_k\cap I|+|I_k\cap I^c|$.\par
\end{proof}

我们发现, 上面的定理和推论都满足可加性, 这说明这些集合具有良好的性质. 而对于一般的集合, 满足这种可加性被称为Carathéodory条件, 其定义如下:\par

\begin{definition}
  设$E\subset \mathbb{R}^n$, 则称$E$满足Carathéodory条件, 当且仅当$\forall T\subset \mathbb{R}^n, m^{*}(T)=m^{*}(T\cap E)+m^{*}(T\cap E^c)$.\par 
\end{definition}

\begin{remark}
  上面的定义说明, 满足Carathéodory条件的集合, 在任意集合上都满足可加性. 这为下面定义可测集提供了一种方式.\par
\end{remark}

\begin{definition}
  设$E\subset \mathbb{R}^n$, 则$E$是可测集当且仅当$E$满足Carathéodory条件.\par
  若$E$是可测的, 则定义其测度为$m(E)=m^{*}(E)$.\par
  记$\mathcal{M}$为所有可测集的集合, $\mathcal{M}$表示可测集族.\par
\end{definition}

\begin{remark}
  由次可加性知, 从定义角度说明一个集合可测只需证$\forall T\subset \mathbb{R}^n, m^{*}(T)\geqslant m^{*}(T\cap E)+m^{*}(T\cap E^c)$.\par
  另一方面, 可以发现测度是外测度的一个限制, 因此测度满足外测度的所有性质, 但反之未必.\par
  此外, 任何集合都可以定义外测度, 但并不是所有集合都可以定义测度.\par
\end{remark}

上面这种条件是针对所有集合的可测性来分析的, 下面我们给出一些定理, 说明一些特殊的集合的可测性可以通过更简单的方法得到.\par

\begin{theorem}
  (余集的可测性) $E$可测等价于$E^c$可测.\par
\end{theorem}

\begin{proof}
  由Carathéodory条件知, $\forall T\subset \mathbb{R}^n, m^{*}(T)=m^{*}(T\cap E)+m^{*}(T\cap E^c)$.\par
  则有$m^{*}(T)=m^{*}(T\cap E^c)+m^{*}(T\cap E)$, 因此$E^c$满足Carathéodory条件, 即$E^c$可测. 反之亦然.\par
\end{proof}

\begin{theorem}
  若$m^{*}(E)=0$, 则$E$是可测的.\par
\end{theorem}

\begin{proof}
  由定义, 只需验证$\forall T\subset \mathbb{R}^n, m^{*}(T)\geqslant m^{*}(T\cap E)+m^{*}(T\cap E^c)$.\par
  由$m^{*}(E)=0$, 有$m^{*}(T\cap E)\leqslant m^{*}(E)=0$.\par
  再考虑$T\cap E\subset E$, $(T\cap E)\cup (T\cap E^c)=T$, 于是$m^{*}(T\cap E^c)\leqslant m^{*}(T)$.\par
  因此有$m^{*}(T)\geqslant m^{*}(T\cap E)+m^{*}(T\cap E^c)$, 即$E$是可测的.\par
\end{proof}

\begin{corollary}
  可列集是可测集.\par
\end{corollary}

\begin{proof}
  由上面的定理, 外测度为0的集合可测. 而可列集的外测度为0, 因此可列集是可测集.\par
\end{proof}

\begin{theorem}
  $E$是可测集, $A\subset E$, $B\subset E^c$当且仅当$m^{*}(A\cup B)=m^{*}(A)+m^{*}(B)$.\par
\end{theorem}

\begin{proof}
  由Carathéodory条件知, $\forall T\subset \mathbb{R}^n, m^{*}(T)=m^{*}(T\cap E)+m^{*}(T\cap E^c)$.\par
  由$T$的任意性, 令$A=T\cap E$, $B=T\cap E^c$, 则有$m^{*}(T)=m^{*}(A)+m^{*}(B)$.\par
  考虑反方向, 设$T=A\cup B$, 由$E$可测, 有$m^{*}(T)=m^{*}(A\cup B)=m^{*}(A)+m^{*}(B)$.\par
  于是有$m^{*}(T\cap E)=m^{*}(A)$, $m^{*}(T\cap E^c)=m^{*}(B)$. 即说明$E$可测.\par
\end{proof}

我们再进一步考虑可测集之间作运算, 是否有类似的结果. 事实上可测集之间的运算是封闭的, 其性质如下:\par

\begin{proposition}
  (集合运算的封闭性) 设$E,F$是可测集, 则:\par
  \begin{itemize}
    \item $E\cup F$是可测集.
    \item $E\cap F$是可测集.
    \item $E\backslash F$是可测集.
  \end{itemize}
\end{proposition}

\begin{proof}
  首先考虑$E\cup F$, 往证$\forall T\subset \mathbb{R}^n, m^{*}(T)=m^{*}(T\cap (E\cup F))+m^{*}(T\cap (E\cup F)^c)$.\par
  注意到$m^{*}(T\cap (E\cup F))=m^{*}(T\cap E)+m^{*}((T\cap F)\backslash E)$, 再由$E$和$F$可测, 有$m^{*}(T)=m^{*}(T\cap E)+m^{*}(T\cap E^c)$, $m^{*}(T\cap E^c)=m^{*}(T\backslash (E\cup F))+m^{*}(T\cap (F\backslash E))$.\par
  于是$\forall T\subset \mathbb{R}^n, m^{*}(T)=m^{*}(T\cap (E\cup F))+m^{*}(T\cap (E\cup F)^c)$.\par
  考虑$E\cap F$, 注意到$E\cup F$已经可测, 因此取余集, $E^c$和$F^c$可测可推出$E\cap F=(E^c\cup F^c)^c$也是可测的.\par
  对于$E\backslash F$, 注意到$E\backslash F=E\cap F^c$, 因此由上面的结论可知$E\backslash F$也是可测的.\par
\end{proof}

\begin{corollary}
  (有限并的封闭性) 设$E_1,\cdots, E_n$是两两不交的可测集, 则:\par
  \begin{itemize}
    \item $m^{*}(T\cap (\bigcup_{i=1}^{n}E_i))=\sum_{i=1}^{n}m^{*}(T\cap E_i)$.
    \item $m(\bigcup_{i=1}^{n}E_i)=\sum_{i=1}^{n}m(E_i)$.
  \end{itemize}
\end{corollary}

\begin{proof}
  由上面的结论, 用归纳法, $\forall T\subset \mathbb{R}^n, m^{*}(T\cap (E_1 \cup E_2))=m^{*}(T\cap E_1)+m^{*}(T\cap E_2)$. 依次作有限次可得.\par
  对于一般情形, 取$T=\mathbb{R}^n$即可.\par
\end{proof}

\begin{corollary}
  (可列并的封闭性) 设$E_1,\cdots, E_n,\cdots $是两两不交的可测集, 则:\par
  \begin{itemize}
    \item $m^{*}(T\cap (\bigcup_{i=1}^{\infty}E_i))=\sum_{i=1}^{\infty}m^{*}(T\cap E_i)$.
    \item $m(\bigcup_{i=1}^{\infty}E_i)=\sum_{i=1}^{\infty}m(E_i)$.
  \end{itemize}
\end{corollary}

\begin{proof}
  考虑$\forall T\subset \mathbb{R}^n, m^{*}(T)=m^{*}(T\cap (\bigcup_{i=1}^{n}E_i))+m^{*}(T\cap (\bigcup_{i=1}^{n}E_i)^c)$.\par
  注意到$m^{*}(T)=m^{*}(T\cap (\bigcup_{i=1}^{n}E_i))= \sum_{i=1}^{n}m^{*}(T\cap E_i)$, $m^{*}(T\cap (\bigcup_{i=1}^{n}E_i)^c)\geqslant m^{*}(T\cap (\bigcup_{i=1}^{\infty}E_i)^c)$.\par
  于是$m^{*}(T)\geqslant \sum_{i=1}^{n}m^{*}(T\cap E_i)+m^{*}(T\cap (\bigcup_{i=1}^{\infty}E_i)^c)$.\par
  令$n\to \infty$, 则有$m^{*}(T)\geqslant \sum_{i=1}^{\infty}m^{*}(T\cap E_i)+m^{*}(T\cap (\bigcup_{i=1}^{\infty}E_i)^c)\geqslant m^{*}(\bigcup_{i=1}^{\infty}(T\cap E_i))+m^{*}(T\cap (\bigcup_{i=1}^{\infty}E_i)^c) = m^{*}(T\cap (\bigcup_{i=1}^{\infty}E_i))+m^{*}(T\cap (\bigcup_{i=1}^{\infty}E_i)^c)$.\par
  因此有$m^{*}(T\cap (\bigcup_{i=1}^{\infty}E_i))=\sum_{i=1}^{\infty}m^{*}(T\cap E_i)$.\par
  对于一般情形, 取$T=\mathbb{R}^n$即可.\par
\end{proof}

这里我们进一步讨论了无穷多个可测集的性质, 更进一步, 我们还可以讨论单调集列的可测性, 并对应得到这些集列的极限的集合的测度的计算方法.\par

\begin{theorem}
  (下连续性) 设$\{E_n\}$是递增集列, 则$\lim_{n \to \infty}E_n$可测, 且$m(\bigcup_{n=1}^{\infty}E_n)=\lim_{n \to \infty}m(E_n)$.\par
  (上连续性) 设$\{E_n\}$是递减集列, 且$\exists E_{n_0}, m(E_{n_0})<\infty$, 则$\lim_{n \to \infty}E_n$可测, 且$m(\bigcap_{n=1}^{\infty}E_n)=\lim_{n \to \infty}m(E_n)$.\par
\end{theorem}

\begin{proof}
  先证明下连续性, 显然递增集列有$m(\bigcup_{n=1}^{\infty}E_n)=m(\lim_{n \to \infty}E_n)$.\par
  取$F_k=E_k \backslash E_{k-1}, \forall k \geqslant 2$, $F_1=E_1$, $E_0=\phi$, 则$\{F_k\}$是两两不交的可测集, 且$\bigcup_{k=1}^{\infty}F_k=\bigcup_{n=1}^{\infty}E_n$.\par
  于是$m(\bigcup_{n=1}^{\infty}E_n)=m(\bigcup_{k=1}^{\infty}F_k) = \sum_{k=1}^{\infty}m(F_k)=\lim_{n \to \infty} m(\bigcup_{k=1}^{n}F_k)=\lim_{n \to \infty}m(E_n)$.\par
  对于上连续性, 我们首先构造$F_{i}=E_1\backslash E_{i}$, 则当$\exists E_{n_0}, m(E_{n_0})<\infty$时有$\{F_i\}$单增可测.\par
  于是由下连续性可得.\par
\end{proof}

\begin{corollary}
  设$\{E_n\}$可测, 且$\sum_{i=1}^{\infty}m(E_i)<\infty$, 则$m(\limsup_{i \to \infty} E_i)=0$.\par
\end{corollary}

\begin{proof}
  $m(\limsup_{i \to \infty} E_i)=\lim_{k \to \infty} m(\bigcup_{i=k}^{\infty}E_i)\leqslant \sum_{i=k}^{\infty}m(E_i)$.\par
\end{proof}

\begin{remark}
  这实际上是Borel- Cantelli引理, 其结果将在各种收敛关系的证明中用到.\par
\end{remark}

最后我们再讨论一些特殊集合的测度.\par

\begin{proposition}
  Cantor集的测度为0.\par
\end{proposition}

\begin{proof}
  Cantor集是通过不断去掉中间的三分之一部分得到的, 因此其外测度为0. 因此测度为0.\par
\end{proof}

\newpage

\section{可测集族}

在讨论了一般的可测集的性质后, 我们更进一步讨论可测集族的性质. 这是一种很自然的想法, 例如讨论定义在区间上的微积分, 我们实际上是讨论了一般区间的一些性质. 因此这里我们也从区间开始研究.\par

\begin{theorem}
  $\mathbb{R}^n$上任意区间$I$可测, 且$m(I)=|I|$.\par
\end{theorem}

\begin{proof}
  易知区间的外测度$m^{*}(I)=|I|$. 同时区间又满足测度的可加性, 因此区间可测.\par
\end{proof}

\begin{corollary}
  $m(\mathbb{R}^n)=\infty$.\par
\end{corollary}

\begin{proof}
  由上可知, $\mathbb{R}^n$可以被任意大的区间所覆盖, 因此其测度为无穷大.\par
\end{proof}

下面我们考虑更一般的可测集, 因此首先我们考虑用更一般的集合来定义外测度.\par

\begin{lemma}
  设$E\subset \mathbb{R}^n$, 则$m^{*}(E)=\inf\{m(G): E\subset G \}$. 其中$G$为开集.\par
\end{lemma}

\begin{proof}
  由外测度的单调性, $m^{*}(E)\leqslant m^{*}(G)=m(G)$.\par
  若$m^{*}(E)=\infty$, 则显然成立.\par
  若$m^{*}(E)<\infty$, 则$\exists \{I_k\}, E\subset \bigcup_{k=1}^{\infty}I_k, m^{*}(E)\leqslant \sum_{k=1}^{\infty}|I_k|<m^{*}(E)+\varepsilon<\infty$.\par
  此时令$G=\bigcup_{k=1}^{\infty}I_k$, 则$E\subset G$, 且$m(G)\leqslant \sum_{k=1}^{\infty}|I_k|<m^{*}(E)+\varepsilon$.\par
  于是有$m^{*}(E)\leqslant m(G)$, 因此$m^{*}(E)=\inf\{m(G): E\subset G \}$.\par
\end{proof}

\begin{lemma}
  开集和闭集均为可测集.\par
\end{lemma}

\begin{proof}
  由上面的引理, 对于开集$G$, 有$m^{*}(G)=\inf\{m(H): G\subset H \}$, 其中$H$为开集.\par
  因此$G$满足Carathéodory条件, 即$G$是可测集.\par
  对于闭集$F$, 注意到$F^c$是开集, 因此由上面的引理知$m^{*}(F^c)=\inf\{m(G): F^c\subset G \}$, 其中$G$为开集.\par
  因此$F^c$满足Carathéodory条件, 即$F^c$是可测集. 于是由余集的可测性知, $F$也是可测集.\par
\end{proof}

\begin{theorem}
  Borel集是可测集.\par
\end{theorem}

\begin{proof}
  Borel集是由开集和闭集通过有限次并、交、补运算得到的集合, 由于开集和闭集都是可测集, 因此由上面的结论知, Borel集也是可测集.\par
\end{proof}

通过上面的讨论, 我们更进一步得到Borel集和可测集的关系, 即Borel集是可测集的一个子集. 事实上, 这是一个真子集. 接下来我们给出可测集的另外一些等价定义.\par

\begin{lemma}
  以下命题等价:\par
  \begin{itemize}
    \item $E$是可测集.
    \item $\forall \varepsilon>0, \exists F\subset E\subset G, m^{*}(G\backslash F)<\varepsilon $, 其中$F$为闭集, $G$为开集.
  \end{itemize}
\end{lemma}

\begin{proof}
  若$E$是可测集, 则$\exists F\subset E, m(E\backslash F)<\frac{\varepsilon}{2}$, 其中$F$为开集.\par
  同理$\exists G^c\subset E^c, m(E^c\backslash G^c)=m(G\backslash E)<\frac{\varepsilon}{2}$, 其中$G^c$为开集.\par
  于是$F\subset E\subset G$, 且$m(G\backslash F)<m(E\backslash F)+m(G\backslash E)<\varepsilon$.\par
  反之, 若$\exists F\subset E\subset G, m^{*}(G\backslash F)<\varepsilon $, 其中$F$为闭集, $G$为开集. 下证$E$可测.\par
  于是$\forall k>0, \exists F_k\subset E\subset G_k, m^{*}(G_k\backslash F_k)<\frac{1}{k}$. 其中$F_k$为闭集, $G_k$为开集.\par
  $m^{*}(E\backslash F_k)\leqslant m^{*}(G_k\backslash F_k)<\frac{1}{k}$, 因此$\lim_{k \to \infty}m^{*}(E\backslash F_k)=0$.\par
  即$m^{*}(E\cap (\bigcup_{k=1}^{\infty}F_k))=0$. $E\cap (\bigcup_{k=1}^{\infty}F_k)$可测.\par
  又有$\bigcup_{k=1}^{\infty}F_k$可测, 于是$E$可测.\par
\end{proof}

\begin{theorem}
  以下命题等价:\par
  \begin{itemize}
    \item $E$是可测集.
    \item $\exists H\supset E, m^{*}(H\backslash E)=0$, 其中$H$为$G_{\delta}$集.
    \item $\exists K\subset E, m^{*}(E\backslash K)=0$, 其中$K$为$F_{\sigma}$集.
    \item $\exists K\subset E\subset H, m(H\backslash K)=0$, 其中$K$为$F_{\sigma}$集, $H$为$G_{\delta}$集.
  \end{itemize}
\end{theorem}

\begin{proof}
  由上面的定理, 我们已经有$E$可测当且仅当$\forall \varepsilon>0, \exists F\subset E\subset G, m^{*}(G\backslash F)<\varepsilon $, 其中$F$为闭集, $G$为开集.\par
  首先考虑$E$可测推出$\exists H\supset E, m^{*}(H\backslash E)=0$, 其中$H$为$G_{\delta}$集.\par
  由$E$可测, $\forall k>0, \exists F\subset E\subset G, m^{*}(G\backslash F)<\frac{1}{k} $, 其中$F$为闭集, $G$为开集.\par
  于是$H=\bigcap_{k=1}^{\infty}G_k$是$G_{\delta}$集, 且$H\supset E$, $m^{*}(H\backslash E)\leqslant m^{*}(G_k\backslash F_k)<\frac{1}{k}$.\par
  令$k\to \infty$, 则$m^{*}(H\backslash E)=0$. 由$H$可测知$E$可测.\par
  类似地, 也可以得到$\exists K\subset E, m^{*}(E\backslash K)=0$, 其中$K$为$F_{\sigma}$集.\par
  反之, 若$\exists H\supset E, m^{*}(H\backslash E)=0$, 其中$H$为$G_{\delta}$集. 下证$E$可测.\par
  由$H$为$G_{\delta}$集, $\exists G_k\supset E, m^{*}(G_k\backslash E)<\frac{1}{k}$, 其中$G_k$为开集.\par
  于是$E\subset G_k$, 且$m^{*}(G_k\backslash E)<\frac{1}{k}$. 令$k \to \infty$, 则$m^{*}(H\backslash E)=0$. 由$H$可测知$E$可测.\par
  类似地, 也可以得到$\exists K\subset E, m^{*}(E\backslash K)=0$, 其中$K$为$F_{\sigma}$集.\par
  下证$E$可测当且仅当$\exists K\subset E\subset H, m(H\backslash K)=0$, 其中$K$为$F_{\sigma}$集, $H$为$G_{\delta}$集.\par
  考虑$E$可测有$\exists H\supset E, m^{*}(H\backslash E)=0$, 其中$H$为$G_{\delta}$集. $\exists K\subset E, m^{*}(E\backslash K)=0$, 其中$K$为$F_{\sigma}$集.\par
  于是有$m^{*}(H\backslash K)\leqslant m^{*}(H\backslash E)+m^{*}(E\backslash K)=0+0=0$. 由$H$为$G_{\delta}$集, $K$为$F_{\sigma}$集, 则$m(H\backslash K)=0$.\par
  下证反方向的命题, 设$\exists K\subset E\subset H, m(H\backslash K)=0$, 其中$K$为$F_{\sigma}$集, $H$为$G_{\delta}$集.\par
  考虑$m^{*}(H\backslash E)\leqslant m^{*}(H\backslash K)=0$, $m^{*}(E\backslash K)\leqslant m^{*}(H\backslash K)=0$.\par
  又有$H$为$G_{\delta}$集, $K$为$F_{\sigma}$集, 于是有$E$可测.\par
\end{proof}

更进一步讨论, 我们有关于可测集族的更显然的性质.\par

\begin{theorem}
  可测集族构成$\sigma$-代数.\par
\end{theorem}

\begin{proof}
  可测集族具有封闭性: 设$E,F$是可测集, 则$E\cup F$, $E\cap F$和$E\backslash F$都是可测集.\par
  再考虑可测集族的补运算, 由余集的可测性知, 若$E$是可测集, 则$E^c$也是可测集.\par
  最后考虑可测集族的可列并和交运算, 由上面的定理知, 可测集族在有限并和交下封闭. 因此在可列并和交下也封闭.\par
  综上所述, 可测集族在补、并、交下封闭, 因此构成$\sigma$-代数.\par
\end{proof}

\newpage

\section{不可测集}

上面所有的讨论都是围绕可测集展开的, 下面我们考虑这样的定义方式是否会导致一些不可测集的出现. 然而在实际问题中, 我们发现尽管不可测集总是存在的, 但其总是不如可测集出现的频繁, 更进一步, 我们还需要讨论不可测集和可测集的数量的比较.\par

\begin{theorem}
  (Vitali集) 存在不可测集.\par
\end{theorem}

\begin{proof}
  首先考虑$\mathbb{R}$上的开区间$E=(0,1)$, 我们下面构造$A\subset E$, 使得$A$不可测.\par

\end{proof}

\newpage

\section{乘积测度}

\btocgroup
\UseTocStyle{chapter}{emph}{toc}
\chapter{可测函数}
\etocgroup

\section{可测函数的定义和性质}

在研究可测函数的性质之前, 我们首先考虑广义实函数的定义:\par

\begin{definition}
  $f:A\rightarrow \mathbb{R}\cup\{-\infty,+\infty\}=\overline{\mathbb{R} }$称为定义在$A$上的广义实函数.\par
\end{definition}

类似地, 我们也有广义实函数的有界性的定义:\par

\begin{definition}
  设$f:A\rightarrow \overline{\mathbb{R} }$是定义在$A$上的广义实函数, 则称$f$是有界的, 当且仅当$\exists M>0, \forall x\in A, |f(x)|<M$.\par
\end{definition}

\begin{remark}
  值得注意的是, 有界性和函数有限并不完全一致, 例如$f(x)=\frac{1}{x}$在$A=(0,1)$上是有界的, 但不是有限的. 另一方面, $f(x)=\chi_{(0,1)}$在$A=(0,1)$上是有限的, 但不是有界的.\par
\end{remark}

事实上, 我们可以给出函数有限的分析定义:\par

\begin{definition}
  设$f:A\rightarrow \overline{\mathbb{R} }$是定义在$A$上的广义实函数, 则称$f$是有限的, 当且仅当$\forall x\in A, f(x)\in \mathbb{R}$.\par
\end{definition}

而对于广义实函数的连续性, 我们也类似地给出定义:\par

\begin{definition}
  设$f$在$E$上有限, 则称$f$在$x_0\in E$处是连续的, 当且仅当$ \forall \varepsilon >0, \exists \delta_{x_0,\varepsilon} >0, \forall x\in E, \rho(x,x_0)<\delta$时有$ |f(x)-f(x_0)|<\varepsilon$.\par
  若$f$在$E$上每一点都是连续的, 则称$f$在$E$上是连续的.\par
  设$f$在$E$上有限, 则称$f$在$E$上是一致连续的, 当且仅当$ \forall \varepsilon >0, \exists \delta_{\varepsilon} >0, \forall x\in E, \rho(x,x_0)<\delta$时有$ |f(x)-f(x_0)|<\varepsilon, \forall x_0 \in E$.\par
\end{definition}

\begin{remark}
  在抽象集合中讨论连续性时, 我们必须要指定讨论的集合范围, 因为在不同的集合上, 同一个函数可能会有不同的连续性. 因此为了方便讨论, 我们下面定义特征函数和简单函数.\par
\end{remark}

\begin{theorem}
  若$E$,$F$为闭集, $f\in C(E)$, $f\in C(F)$, 则$f\in C(E \cup F)$.\par
\end{theorem}

\begin{proof}
  若$x_0\in E\cup F$, 则$\forall \varepsilon >0, \exists \delta_{1} >0$, 使得$\forall x\in E, \rho(x,x_0)<\delta_{1}$时有$|f(x)-f(x_0)|<\varepsilon$.\par
  同理, $\forall \varepsilon >0, \exists \delta_{2} >0$, 使得$\forall x\in F, \rho(x,x_0)<\delta_{2}$时有$|f(x)-f(x_0)|<\varepsilon$.\par
  于是取$\delta = \min\{\delta_{1},\delta_{2}\}$, 则$\forall x\in E\cup F, \rho(x,x_0)<\delta$时有$|f(x)-f(x_0)|<\varepsilon$.\par
  若$x_0\in E\backslash F$, 则$\forall \varepsilon >0, \exists \delta_{1} >0$, 使得$\forall x\in E, \rho(x,x_0)<\delta_{1}$时有$|f(x)-f(x_0)|<\varepsilon$.\par
  同时有$\exists \delta_{3} >0$, 使得$O(x_0,\delta_{3})\subset F^c$.\par
  于是取$\delta = \min\{\delta_{1},\delta_{3}\}$, 则$\forall x\in E\cup F, \rho(x,x_0)<\delta$时有$x\in E\backslash F$且$\rho(x,x_0)<\delta_{1}$,$|f(x)-f(x_0)|<\varepsilon$.\par
  对于$x_0\in F\backslash E$, 类似可证.\par
\end{proof}

\begin{definition}
  设$E$是集合, 则称$E$的特征函数为$\chi_E(x)=\begin{cases}
    1, & x\in E \\
    0, & x\notin E
  \end{cases}$.\par
  设$E_i$是一集列, 则称$E_i$的简单函数为$f(x)=\sum_{i=1}^{n}c_i\chi_{E_i}(x)$, 其中$c_i\in \mathbb{R}$, 且$E_i$互不交.\par
  特别地, 若$E_i$均为区间, 则称$f$是阶梯函数.\par
\end{definition}

根据上面的讨论, 我们有一类特殊的例子来表示广义实函数的连续性:\par

\begin{example}
  $f=\chi_{\mathbb{Q}}(x)$在$\mathbb{Q}$上是连续的, 但在$\mathbb{R}$上是处处不连续的.\par
\end{example}

基于此我们给出可测函数的定义:\par

\begin{definition}
  设$E$是集合, $f$是定义在$E$上的广义实函数, 则称$f$在$E$上是可测的, 当且仅当$\forall a\in \mathbb{R}, E[f> a]$是可测集.\par
  若$f$在$E$上是可测的, 则称$f$是$E$上的可测函数, 记为$f\in \mathcal{M}(E)$.\par
\end{definition}

\begin{theorem}
  下面函数均为可测函数:\par
  \begin{itemize}
    \item 可测集的特征函数.\par
    \item 简单函数.\par
    \item 可测集上的连续函数.\par
    \item 可测集上的单调函数.\par
    \item 零测集上的函数.\par
  \end{itemize}
\end{theorem}

\begin{proof}
  对于可测集的特征函数, 由定义可知, $\forall a\in \mathbb{R}, E[\chi_E > a] = E$或$E^c$, 因此是可测的.\par
  对于简单函数, 设$f(x)=\sum_{i=1}^{n}c_i\chi_{E_i}(x)$, 则$\forall a\in \mathbb{R}, E[f>a]=\bigcup_{c_i>a}E_i$, 因此是可测的.\par
  对于可测集上的连续函数, 由连续函数的性质知, $\forall a\in \mathbb{R}, E[f>a]$是开集或闭集, 因此是可测的.\par
  对于可测集上的单调函数, 类似地, $\forall a\in \mathbb{R}, E[f>a]$是开集或闭集, 因此是可测的.\par
  对于零测集上的函数, 由于零测集上的函数在几乎处处上是常数函数, 因此也是可测的.\par
\end{proof}

事实上可测函数还有一些等价的定义, 我们下面来证明这些定义是等价的.\par

\begin{theorem}
  设$f$是定义在$E$上的广义实函数, 则下列命题等价:\par
  \begin{itemize}
    \item $f$在$E$上是可测的.
    \item $\forall a\in \mathbb{R}, E[f>a]$是可测集.
    \item $\forall a\in \mathbb{R}, E[f\geqslant a]$是可测集.
    \item $\forall a\in \mathbb{R}, E[f<a]$是可测集.
    \item $\forall a\in \mathbb{R}, E[f\leqslant a]$是可测集.
  \end{itemize}
\end{theorem}

\begin{proof}
  由定义知, $f$在$E$上是可测的当且仅当$\forall a\in \mathbb{R}, E[f>a]$是可测集.\par
  首先考虑$E[f\leqslant a]$是$E[f>a]$的余集, 因此$E[f\leqslant a]$的可测性和$E[f>a]$是等价的.\par
  下面考虑$E[f<a]=\bigcup_{k=1}^{\infty}E[f\leqslant a-\frac{1}{k}]$, 而对于每个$a\in \mathbb{R}$, $E[f\leqslant a-\frac{1}{k}]$是可测集, 因此$\bigcup_{k=1}^{\infty}E[f\leqslant a-\frac{1}{k}]$也是可测集.\par
  于是类似地, 由$E[f\geqslant a]$是$E[f<a]$的可测性是等价的知$E[f\geqslant a]$可测也是成立的.\par
\end{proof}

下面我们利用可测函数的定义来讨论一些初等和基本的性质.\par

\begin{proposition}
  若$f$在$E$上是可测的, 则$\forall a\in \overline{\mathbb{R}}, E[f=a]$可测.\par
\end{proposition}

\begin{proof}
  对于$a \in \mathbb{R}$, 我们有$E[f\geqslant a]\cap E[f\leqslant a]=E[f=a]$, 于是可测.\par
  对于$a = +\infty$, 则$E[f=+\infty]=\bigcap_{n=1}^{\infty}E[f\geqslant n]$, 由于$E[f\geqslant n]$是可测集, 因此$E[f=+\infty]$也是可测的.\par
  对于$a = -\infty$, 则$E[f=-\infty]=\bigcap_{n=1}^{\infty}E[f\leqslant -n]$, 由于$E[f\leqslant -n]$是可测集, 因此$E[f=-\infty]$也是可测的.\par
\end{proof}

\begin{remark}
  上面的结果是充分的, 但不是必要的. 即$\forall a \in \mathbb{R}, E[f=a]$可测不蕴含$f$在$E$上是可测的.\par
\end{remark}

\begin{proposition}
  $f\in \mathcal{M}(E)$当且仅当$\forall E_1\subset E, f\in \mathcal{M}(E_1)$, 其中$E_1$可测.\par
\end{proposition}

\begin{proof}
  由于$\forall a \in \mathbb{R}$, $E_1[f>a]=E_1\cap E[f>a]$可测, 因此$E_1[f>a]$可测.\par
\end{proof}

\begin{proposition}
  记$E=\bigcup_{i=1}^{\infty}E_i$, $\{E_i\}$互不交可测, 则$f\in \mathcal{M}(E)\Leftrightarrow f\in \mathcal{M}(E_i), \forall i$.\par
\end{proposition}

\begin{proof}
  由$E=\bigcup_{i=1}^{\infty}E_i$, $\forall a \in \mathbb{R}$, 有$E[f>a]=\bigcup_{i=1}^{\infty}E_i[f>a]$, 由于$\{E_i\}$互不交可测, 因此$E[f>a]$是可测的当且仅当$\forall i, E_i[f>a]$是可测的.\par
\end{proof}

\begin{remark}
  上面的结果说明了可测函数具有良好的性质, 即可测函数在可测集上的限制仍然是可测的.\par
\end{remark}

\begin{theorem}
  若$\{f_n\}\in \mathcal{M}(E)$, 则$\sup \{f_n\}$, $\inf \{f_n\}$, $\limsup_{n \to \infty} f_n$, $\liminf_{n \to \infty} f_n$, $\lim_{n \to \infty} f_n$均在$E$上可测.\par
\end{theorem}

\begin{proof}
  事实上我们只需要证明$\sup \{f_n\}$和$\inf \{f_n\}$在$E$上可测, 上下极限事实上是由$\sup \{f_n\}$和$\inf \{f_n\}$来定义的, 因此其可测性在上下确界可测时自然成立.\par
  首先考虑$\sup \{f_n\}$, 由定义知, $\forall a \in \mathbb{R}, E[\sup \{f_n\}\leqslant a]=\bigcap_{n=1}^{\infty}E[f_n\leqslant a]$, 由于$\{f_n\}\in \mathcal{M}(E)$, 因此$E[\sup \{f_n\}\leqslant a]$也是可测的.\par
  类似地, 我们有$E[\inf \{f_n\}\geqslant a]=\bigcap_{n=1}^{\infty}E[f_n\geqslant a]$, 其可测性显然.\par
  于是$\sup \{f_n\}$和$\inf \{f_n\}$在$E$上均可测.\par
  由于$\limsup_{n \to \infty} f_n = \inf_{N\geqslant 1} \sup_{n\geqslant N} \{f_n\}$, $\liminf_{n \to \infty} f_n = \sup_{N\geqslant 1} \inf_{n\geqslant N} \{f_n\}$.\par
  其可测性显然, 因为$\sup$和$\inf$的可测性已经证明.\par
  最后考虑$\lim_{n \to \infty} f_n$存在时有$\lim_{n \to \infty} f_n = \limsup_{n \to \infty} f_n = \liminf_{n \to \infty} f_n$, 由于$\limsup_{n \to \infty} f_n$和$\liminf_{n \to \infty} f_n$均在$E$上可测. 因此其可测性也显然.\par
\end{proof}

\begin{remark}
  上面的结果事实上反映了可测函数列的极限的可测性, 即可测函数列的极限仍然是可测函数.\par
  同时注意到这里我们选择的证明方式是利用可测函数的等价定义构造可测集的可列交得到的, 这和我们证明可测函数的等价定义时的想法是类似的.\par
\end{remark}

下面我们考虑简单函数列和可测函数的关系, 这里我们首先考虑连续函数列的情形.\par

\begin{lemma}
  连续函数列的极限在可测集上可测.\par
\end{lemma}

\begin{proof}
  设$f_n$是定义在$E$上的连续函数列, 且$f_n\to f$, 则$\forall a\in \mathbb{R}$, $E[f_n>a]$是开集, 因此是可测的.\par
  由于$\bigcap_{k=1}^{\infty}E[f_n> a]=E[f>a]$, 因此$E[f>a]$也是可测的.\par
  于是$f$在$E$上是可测的.\par
\end{proof}

\begin{corollary}
  简单函数列的极限在可测集上可测.\par
\end{corollary}

\begin{proof}
  设$f_n$是定义在$E$上的简单函数列, 且$f_n\to f$, 则$\forall a\in \mathbb{R}$, $E[f_n>a]$是可测集, 因此是可测的.\par
  由于$\bigcap_{k=1}^{\infty}E[f_n> a]=E[f>a]$, 因此$E[f>a]$也是可测的.\par
  于是$f$在$E$上是可测的.\par
\end{proof}

下面考虑简单函数列的极限, 我们有下面的结果:\par

\begin{theorem}
  设$f\geqslant 0$, $f\in \mathcal{M}(E)$, 则$\exists \{\phi_n\}$, $\phi_n$为$E$上单增简单函数列满足$\lim_{n \to \infty} \phi_n = f$.\par
\end{theorem}

\begin{proof}
  $\forall n \in \mathbb{N}$, 将$[0,+\infty]$拆成若干小区间: $[0,\frac{1}{2^n}), [\frac{1}{2^n}, \frac{2}{2^n}), \cdots, [\frac{(n2^n-1)}{2^n},n),[n,+\infty]$.
  同时也有$E=E[0\leqslant f<\frac{1}{2^n}]\cup E[\frac{1}{2^n}\leqslant f<\frac{2}{2^n}]\cup \cdots \cup E[\frac{(n2^n-1)}{2^n}\leqslant f<n]\cup E[f\geqslant n]$.\par
  于是我们可以定义$\phi_n(x)=\sum_{k=0}^{n} \frac{k}{2^n}\chi_{E[\frac{k}{2^n}\leqslant f<\frac{k+1}{2^n}]}(x)$. 下面说明这就是我们要求的简单函数列.\par
  首先$\phi_n$是单增的, 因为$\forall x\in E$, $\phi_n(x)$是单调递增的.\par
  其次$\phi_n$是简单函数, 因为$\phi_n(x)$是有限个特征函数的线性组合.\par
  最后考虑$\lim_{n \to \infty} \phi_n(x)$, 由于$\phi_n(x)$是单调递增的, 且$\phi_n(x)\leqslant f(x)$, 因此$\lim_{n \to \infty} \phi_n(x) = f(x)$.\par
  于是$\phi_n$是满足要求的单增简单函数列.\par
\end{proof}

下面我们讨论一般函数的情形, 为了方便起见, 我们这里定义一般函数的非负部分和非正部分:\par

\begin{definition}
  设$f$是定义在$E$上的广义实函数, 则称$f$的正部为$f^+=\max\{f,0\}$, 负部为$f^-=\max\{-f,0\}$.\par
\end{definition}

\begin{remark}
  对于函数的正部和负部, 我们有一些基本结论, 这里不再证明.\par
  \begin{itemize}
    \item $f=f^+-f^-$.
    \item $|f|=f^++f^-$.
    \item $f^+\geqslant 0$, $f^-\geqslant 0$.
  \end{itemize}
\end{remark}

\begin{corollary}
  设$f\in \mathcal{M}(E)$, 则$\exists \{\phi_n\}$, $\phi_n$为$E$上简单函数列满足$\lim_{n \to \infty} \phi_n = f$.\par
\end{corollary}

\begin{proof}
  由上面的结果, 把$f$分解为正部和负部, 即$f=f^+-f^-$, 其中$f^+=\max\{f,0\}$, $f^-=\max\{-f,0\}$.\par
  由于$f^+$和$f^-$均为非负可测函数, 因此存在单增简单函数列$\{\phi_n^{+}\}$和$\{\phi_n^{-}\}$逼近$f^+$和$f^-$.\par
\end{proof}

\begin{remark}
  上面的结果说明了可测函数可以用一列单增的简单函数列逼近, 一方面这是和一般的微积分中的结果是类似的, 另一方面这也为下面研究可测函数提供了重要支持.\par
\end{remark}

下面我们来讨论可测函数的运算封闭性, 实际上这是由上面简单函数的逼近得到的.\par

\begin{theorem}
  设$f,g\in \mathcal{M}(E)$, 则$f+g$, $f-g$, $fg$, $\frac{f}{g}$在$E$上有意义时均可测.\par
\end{theorem}

\begin{proof}
  由于$f$和$g$均为可测函数, 因此$\exists \{\phi_n\}$和$\{\psi_n\}$, $\phi_n$和$\psi_n$均为$E$上单增简单函数列满足$\lim_{n \to \infty} \phi_n = f$, $\lim_{n \to \infty} \psi_n = g$.\par
  于是考虑$f+g$, $f-g$, $fg$, $\frac{f}{g}$的情况:\par
  \begin{itemize}
    \item 对于$f+g$, $\exists \{\phi_n+\psi_n\}$, $\phi_n+\psi_n$为$E$上单增简单函数列满足$\lim_{n \to \infty} (\phi_n+\psi_n) = f+g$.\par
    \item 对于$f-g$, $\exists \{\phi_n-\psi_n\}$, $\phi_n-\psi_n$为$E$上单增简单函数列满足$\lim_{n \to \infty} (\phi_n-\psi_n) = f-g$.\par
    \item 对于$fg$, $\exists \{\phi_n\cdot\psi_n\}$, $\phi_n\cdot\psi_n$为$E$上单增简单函数列满足$\lim_{n \to \infty} (\phi_n\cdot\psi_n) = fg$.\par
    \item 对于$\frac{f}{g}$, 若$\forall x\in E, g(x)\neq 0$, 则$\exists \{\frac{\phi_n}{\psi_n}\}$, $\frac{\phi_n}{\psi_n}$为$E$上单增简单函数列满足$\lim_{n \to \infty} (\frac{\phi_n}{\psi_n}) = \frac{f}{g}$.\par
  \end{itemize}
  于是$f+g$, $f-g$, $fg$, $\frac{f}{g}$在$E$上均可测.\par
\end{proof}

\begin{remark}
  上面的结果说明了可测函数在几乎处处有限的情形下构成线性空间, 即可测函数的和、差、积和商(除数不为零时)仍然是可测函数.\par
\end{remark}

最后我们给出一个关于简单函数列和函数的可测性间关系的最简洁的结果, 这个结果可以用来判断函数的可测性.\par

\begin{theorem}
  $f\in \mathcal{M}(E)$当且仅当$\exists \{\phi_n\}$, $\phi_n$为$E$上简单函数列满足$\lim_{n \to \infty} \phi_n = f$.\par
\end{theorem}

\begin{proof}
  由上结果可知, 若$f\in \mathcal{M}(E)$, 则$\exists \{\phi_n\}$, $\phi_n$为$E$上单增简单函数列满足$\lim_{n \to \infty} \phi_n = f$.\par
  下证反方向的命题, 设$\exists \{\phi_n\}$, $\phi_n$为$E$上简单函数列满足$\lim_{n \to \infty} \phi_n = f$.\par
  由于$\phi_n$是简单函数, 因此$\forall a\in \mathbb{R}$, $E[\phi_n>a]$是可测集, 因此$\bigcap_{k=1}^{\infty}E[\phi_n>a]$也是可测集.\par
  于是$\forall a\in \mathbb{R}$, $E[f>a] = \bigcap_{k=1}^{\infty}E[\phi_n>a]$是可测集.\par
  由于$\forall a\in \mathbb{R}$, $E[f>a]$是可测集, 因此$f$在$E$上是可测的.\par
\end{proof}

\newpage

\section{可测函数的收敛性}

在讨论函数性质时, 我们自然想到普通微积分中实函数的收敛, 因此我们现在对函数的收敛作如下定义:\par

\begin{definition}
  设$f_n,f$是定义在$E$上的函数, 
  \begin{itemize}
    \item 称$f_n$逐点收敛到$f$, 当且仅当$\forall x\in E, \lim_{n \to \infty}f_n(x)=f(x)$.\par
    \item 称$f_n$一致收敛到$f$, 当且仅当$\forall \varepsilon>0, \exists N\in \mathbb{N}, \forall n\geqslant N, |f_n(x)-f(x)|<\varepsilon$.\par
    \item 称$f_n$几乎处处收敛到$f$, 当且仅当$\exists E_0\subset E, m(E_0) = 0, \forall x\in E\backslash E_0, \lim_{n \to \infty}f_n(x)=f(x)$.\par
    \item 称$f_n$近乎一致收敛到$f$, 当且仅当$\forall \varepsilon>0, \forall \delta > 0, \exists E_0\subset E, m(E_0) < \delta, \exists N\in \mathbb{N}, \forall n\geqslant N, \forall x\in E\backslash E_0, |f_n(x)-f(x)|<\varepsilon$.\par
  \end{itemize}
\end{definition}

\begin{example}
  若$f_n(x)=x^n$, $f(x)=0$, 则$f_n(x)$近乎一致收敛到$f$.\par
  这是由于$\forall \delta>0$, 取$E_0 =\left[1-\frac{\delta}{2},1\right]$, 则$m(E_0)=\frac{\delta}{2}<\delta$, $\forall x\in E\backslash E_0$, $|f_n(x)-f(x)|=|x^n|<\varepsilon$.\par
  此时只需要取$N=\left|\frac{\varepsilon}{\log \left(1-\frac{\delta}{2}\right)}\right|$, 使得$\forall n\geqslant N$, $|x^n|<\varepsilon$.\par
\end{example}

\begin{remark}
  这里$N$是和$\delta$有关的, 但不能和$x$有关, 否则就不再是一致收敛了.\par
\end{remark}

\begin{theorem}
  下列命题等价:\par
  \begin{itemize}
    \item $\exists E_0\subset E, m(E_0) = 0, \forall x\in E\backslash E_0, \lim_{n \to \infty}f_n(x)=f(x)$.
    \item $\lim_{n\to \infty}m(E[f_n \nrightarrow f])=0$.
    \item $\forall x\in E[f_n \nrightarrow f], f_n(x)\nrightarrow f(x)$
  \end{itemize}
\end{theorem}

\begin{lemma}
  $E[f_n \nrightarrow f] = \bigcup_{\varepsilon} \bigcap_{N=1}^{+\infty} \bigcup_{n=N}^{+\infty}E\left[\left|f_n-f\right|\geqslant \varepsilon\right] = \bigcup_{k=1}^{+\infty} \bigcap_{N=1}^{+\infty} \bigcup_{n=N}^{+\infty}E\left[\left|f_n-f\right|\geqslant \frac{1}{k}\right]$.
\end{lemma}

\begin{proof}
  考虑$E[f_n \nrightarrow f]$的定义:\par
  $\exists \varepsilon_0>0, \forall N\in \mathbb{N}, \exists n_0\geqslant N, \left|f_n(x)-f(x)\right|\geqslant \varepsilon_0$.\par
  这说明$\exists \varepsilon_0>0, \forall N\in \mathbb{N}, \exists n_0\geqslant N, x\in E\left[\left|f_n-f\right|\geqslant \varepsilon_0\right]$.\par
  于是有$x\in \bigcup_{\varepsilon} \bigcap_{N=1}^{+\infty} \bigcup_{n=N}^{+\infty}E\left[\left|f_n-f\right|\geqslant \varepsilon_0\right]$.\par
  更进一步考虑$\varepsilon$的任意性, 我们有$x \in \bigcup_{k=1}^{+\infty} \bigcap_{N=1}^{+\infty} \bigcup_{n=N}^{+\infty}E\left[\left|f_n-f\right|\geqslant \frac{1}{k}\right]$.\par
\end{proof}

\begin{remark}
  上面的这种方法在后续的Lebesgue积分中也会用到, 这种方法实际上是利用了$\varepsilon$的任意性来构造一个新的集合, 使得这个集合的测度为0. 但同时考虑到$\varepsilon$是任意的, 因此我们选择$\frac{1}{k}$代替依旧是成立的.\par
\end{remark}

\begin{lemma}
  (Borel-Cantelli) 设$\{A_n\}$是一集列, 则$\sum_{n=1}^{\infty}m(A_n)<\infty \Rightarrow m(\limsup_{n \to \infty}A_n)=0$.
\end{lemma}

\begin{proof}
  首先考虑$e_k$和$E$均可测, 则有下面的等价关系:\par
  $E\backslash \left(\limsup_{k \to \infty}e_k \right)= \liminf_{k\to \infty}\left(E\backslash e_k\right)$.\par
  事实上我们有$\left(\limsup_{k\to \infty}e_k\right)^{c}=\left(\bigcap_{N=1}^{\infty}\left(\bigcup_{k=N}^{\infty}e_k\right)\right)^{c}=\bigcup_{N=1}^{\infty}\left(\bigcup_{k=N}^{\infty}e_k\right)^{c} = \bigcup_{N=1}^{\infty}\bigcap_{k=N}^{\infty}e_k^c$.\par
  下面再考虑$m(E_k)<\frac{1}{2^k}, \forall k$, 则有$m\left(\bigcup_{k=N}^{\infty}E_k\right)\leqslant \frac{1}{2^{N-1}}$. 记$F_N = \bigcup_{k=N}^{\infty}E_k$, 则$m\left(\bigcap_{N=1}^{\infty}F_N\right)\leqslant m\left(F_N\right)\to 0$.\par
\end{proof}

下面我们讨论近乎一致收敛, 几乎处处收敛的关系:\par

\begin{theorem}
  若$f_n,f$在$E$上可测, $f_n$近乎一致收敛到$f$, 则$f_n$几乎处处收敛到$f$.\par
\end{theorem}

\begin{proof}
  $f_n$近乎一致收敛到$f$, 则有$\forall k >0, \exists {e_k}, m(e_k)<\frac{1}{k}, f_n \Rightarrow f \text{on} E\backslash e_k$, 其中$e_k$可测.\par
  往证$m\left(\bigcup_{k=1}^{\infty}\bigcap_{N=1}^{\infty}\bigcup_{n=N}^{\infty} E\left[\left|f_n-f\right|\geqslant \frac{1}{k}\right]\right)=0$. 令$E_0=\limsup_{k \to \infty}e_k$, 则$E\left[f_n\nrightarrow f\right]\subset E_0$.\par
  由Borel-Cantelli引理, 我们有$\sum_{n=1}^{\infty}m\left(E\left[\left|f_n-f\right|\geqslant \frac{1}{k}\right]\right)<\infty$. \par
  因此$m\left(\limsup_{n \to \infty}E\left[\left|f_n-f\right|\geqslant \frac{1}{k}\right]\right)=0$.\par
\end{proof}

\begin{theorem}
  (Egoroff) 设$m(E)<\infty$, 若$f_n$在$E$上几乎处处收敛到$f$, 则$f_n$近乎一致收敛到$f$.\par
\end{theorem}

\begin{proof}
  由$f_n \rightarrow f \text{ a.e.} on E$知, $m\left(\bigcup_{k=1}^{\infty}\bigcap_{N=1}^{\infty}\bigcup_{n=N}^{\infty}E\left[|f_n-f|\geqslant \frac{1}{k}\right]\right)=0$.\par
  记$F_k=\bigcap_{N=1}^{\infty}\bigcup_{n=N}^{\infty}E\left[|f_n-f|\geqslant \frac{1}{k}\right]$, 则$m(F_k)=0, \forall k >0$. 由$m(E)<\infty$有$\lim_{k\to \infty}m(F_k) = m(\lim_{k\to \infty}F_k)=0$.\par
  即$\forall k > 0, \forall \delta >0, \exists k_{\delta}>0, \forall k>k_{\delta}, m(F_{k_{\delta}})<\frac{\delta}{2^{k_{\delta}}}$, 于是$m\left(\bigcup_{k\geqslant k_{\delta}}F_k\right)<\sum_{N=1}^{\infty}\frac{\delta}{2^{k_{\delta}}}=\delta$.\par
  令$e=\bigcup_{n\geqslant k_{\delta}}F_k$, 则$m(e)<\delta$, $\forall x\in E\backslash e, |f_n(x)-f(x)|<\frac{1}{k}$, 这说明$f_n$近乎一致收敛到$f$.\par 
\end{proof}

根据上面的讨论, 我们知道近乎一致收敛有下面的表述方法:
$$\forall \delta >0, \forall \varepsilon >0, \exists E_0\subset E, m(E_0)<\delta, \exists N>0, \forall n\geqslant N, |f_n(x)-f(x)|<\varepsilon, \forall x\in E\backslash E_0.$$

事实上, 我们知道这表示$|f_n(x)-f(x)|\geqslant \varepsilon$的集合的测度应当很小, 而又考虑到$\delta$是任意取的, $N$只与$\delta$有关, 于是我们可以作下面的定义:
$$\forall \delta >0, \forall \varepsilon >0, \exists N, n\geqslant N, m\left(E\left[|f_n-f|\geqslant \varepsilon\right]\right)<\delta.$$

更进一步, 这可以写成
$\forall \varepsilon >0, \lim_{n \to \infty}m\left(E\left[|f_n-f|\geqslant \varepsilon\right]\right) = 0.$

这就有了下面依测度收敛的定义:\par
\begin{definition}
  设$f_n,f$是定义在$E$可测且几乎处处有限的函数, 则$f_n$依测度收敛到$f$, 当且仅当$\forall \varepsilon >0, \lim_{n \to \infty}m\left(E\left[|f_n-f|\geqslant \varepsilon\right]\right) = 0$. 记作$f_n\Rightarrow f$.
\end{definition}

对于这种收敛, 我们继续讨论和其他收敛的关系, 于是有下面的定理:\par

\begin{theorem}
  (Lebesgue) 设$m(E)<\infty$, 若$f_n$几乎处处收敛到$f$, 则$f_n$依测度收敛到$f$.\par
\end{theorem}

\begin{proof}
  由$f_n$几乎处处收敛到$f$, 则$\exists E_0\subset E, m(E_0)=0, \forall x\in E\backslash E_0, \lim_{n \to \infty}f_n(x)=f(x)$.\par
  于是有$\forall \varepsilon >0, \exists N>0, \forall n\geqslant N, |f_n(x)-f(x)|<\varepsilon$, $\forall x\in E\backslash E_0$.\par
  因此$m\left(E\left[|f_n-f|\geqslant \varepsilon\right]\right)\leqslant m(E_0)=0$, 这说明$f_n$依测度收敛到$f$.\par
\end{proof}

\begin{theorem}
  (Riesz) 若$f_n$依测度收敛到$f$, 则$\exists f_{n_k}$, $f_{n_k}$几乎处处收敛到$f$.\par
\end{theorem}

\begin{proof}
  对于依测度收敛的函数列$f_n$, 总可以取$k>0$使得$e_k=E\left[|f_{n_k}-f|\geqslant \varepsilon\right]$满足$m(e_k)< \frac{1}{2^k}$. 于是$\sum_{k=1}^{\infty}m(e_k)=1<\infty$.\par
  由Borel-Cantelli引理, 我们有$m\left(\limsup_{k \to \infty}e_k\right)=0$, 令$E_0=\limsup_{k \to \infty}e_k$, 则$\forall x \in E\backslash E_0, \exists k_0, \forall k\geqslant k_0, |f_{n_k}(x)-f(x)|<\varepsilon$.\par
  这说明$f_{n_k}$几乎处处收敛到$f$.\par
\end{proof}

\begin{remark}
  几乎处处收敛和依测度收敛不是等价的条件, 例如下面两个例子:\par
  令$f_n(x)=\chi_{\left(\right.0,n\left.\right]}$, 则$f_n(x)$几乎处处收敛到$f(x)=\chi_{\left(\right.0,+\infty\left.\right)}$, 但$f_n$不依测度收敛到$f$, 这是因为我们可以验证$\lim_{n \to \infty}m\left(E\left[|f_n-f|\geqslant 1\right]\right) = +\infty$.\par
  令$f_n(x)$按下面的方法排列: $f_1(x)=\chi_{\left[\right.0,\frac{1}{2}\left.\right]}$, $f_2(x)=\chi_{\left(\right.\frac{1}{2},1\left.\right]}$, $f_3(x)=\chi_{\left[\right.0,\frac{1}{4}\left.\right]}$, $f_4(x)=\chi_{\left(\right.\frac{1}{4},\frac{1}{2}\left.\right]}$,$\cdots$, 则$f_n(x)$依测度收敛到$f(x)=\chi_{\left[\right.0,1\left.\right]}$, 但$f_n$处处不收敛到$f$.\par
\end{remark}

基于上面的结果, 我们下面讨论依测度收敛的性质:\par

\begin{example}
  若$f_n\Rightarrow f$, $f_n\Rightarrow g$, 则$f=g$几乎处处成立.\par
\end{example}

\begin{example}
  若$f_n\Rightarrow f$, $g_n\Rightarrow g$, $f_n\geqslant g_n$在$E$上几乎处处成立, 则$f\geqslant g$在$E$上几乎处处成立.\par
\end{example}

\begin{example}
  若$f_n\Rightarrow f$, $a\in \mathbb{R}$, 则$af_n\Rightarrow af$.\par  
  若$f_n\Rightarrow f$, $g_n \Rightarrow g$, 则$f_n+g_n\Rightarrow f+g$.\par
\end{example}

\begin{corollary}
  若$f_n\Rightarrow 0$, $g_n \Rightarrow 0$, 则$f_ng_n\Rightarrow 0$.\par
\end{corollary}

\begin{proof}
  往证: $\forall \varepsilon >0, \lim_{n \to \infty}m\left(E\left[|f_ng_n|\geqslant \varepsilon\right]\right) = 0$.\par
  记$E_n=E\left[|g_n|>1\right]$, 则$E_n^c=E\left[|g_n|\leqslant 1\right]$.\par
  于是$E\left[|f_ng_n|\geqslant \varepsilon\right]\subset E_n\left[|f_ng_n|\geqslant \varepsilon\right]\cup E_n^c\left[|f_ng_n|\geqslant \varepsilon\right]\subset E_n\cup E\left[|f_n|\geqslant \varepsilon\right]$.\par
  由$f_n,g_n \Rightarrow 0$, $\exists N_1,N_2$使得$n\geqslant N_1, m(E\left[|f_n|\geqslant \varepsilon\right])<\frac{\delta}{2}$, $n\geqslant N_2, m(E\left[|g_n|\geqslant 1\right])<\frac{\delta}{2}$.\par
  取$N = \max\{N_1,N_2\}$, 则$\forall n\geqslant N$, $m(E\left[|f_ng_n|\geqslant \varepsilon\right])<\frac{\delta}{2}+\frac{\delta}{2}=\delta$.\par
  这说明$\lim_{n \to \infty}m\left(E\left[|f_ng_n|\geqslant \varepsilon\right]\right) = 0$. 即$f_ng_n \Rightarrow 0$.\par
\end{proof}

\begin{corollary}
  若$f_n\Rightarrow f$, $g_n = g$有界可测, 则$f_ng\Rightarrow fg$.\par
\end{corollary}

\begin{proof}
  往证: $\forall \varepsilon >0, \lim_{n \to \infty}m\left(E\left[|f_ng-fg|\geqslant \varepsilon\right]\right) = 0$.\par
  由$f_n\Rightarrow f$, $\exists N, \forall n\geqslant N, m(E\left[|f_n-f|\geqslant \frac{\varepsilon}{M}\right])<\delta$.\par
  注意到$E\left[|f_ng-fg|\geqslant \varepsilon\right]\subset  E\left[|f_n-f|\geqslant \frac{\varepsilon}{M}\right]$,其中$M=\sup |g_n|=\sup |g|$.\par
  于是$m\left(E\left[|f_ng-fg|\geqslant \varepsilon\right]\right)\leqslant m\left(E\left[|f_n-f|\geqslant \frac{\varepsilon}{M}\right]\right)<\delta$. 即$f_ng\Rightarrow fg$.\par
\end{proof}

\begin{remark}
  上面的若干例子说明了依测度收敛的函数列在数乘和加法下是封闭的, 但同时注意到第二个推论中的有界性是必要的, 例如下面这个例子:\par
  令$f_n(x)=\chi_{\left(\right.0,n\left.\right]} \frac{1}{x}$, $g(x)=x$, 则$f_n\Rightarrow 0$, 但$f_n\cdot g = \chi_{\left(\right.0,n\left.\right]}$, $f\cdot g = 1$. 此时并没有$\chi_{\left(\right.0,n\left.\right]}\Rightarrow \chi_{\left(0,\infty\right)}$.\par
\end{remark}

\newpage

\section{可测函数的连续性}

上面讨论的许多收敛都是在可测函数的基础上进行的, 现在我们讨论可测函数的连续性, 这一点是必须的, 这是因为后续我们在建立可积函数, 可测函数与联系函数的关系时会更进一步讨论连续的特点. \par

\begin{lemma}
  若$E$是一可测集, 则$C(E)\subset \mathcal{M} (E)$.
\end{lemma}

\begin{proof}
  由$E$是可测集, 因此$E[f>a]=E\cap f^{-1}\left(a,\infty\right), \forall a \in \mathbb{R}$. \par
  由连续映射的定义知, $f^{-1}\left(a,\infty\right)$是开集. 于是$E\cap f^{-1}\left(a,\infty\right)$可测.\par
  即$E[f>a],\forall a \in \mathbb{R}$是可测的. 于是$f$是可测函数.\par
\end{proof}

上面这个定理说明了连续函数在可测集上是可测的, 下面我们为了得到更强结果的连续性条件, 我们首先考虑下面的一系列问题.\par

\begin{lemma}
  设$A,B$均为闭集, $A\cap B=\Phi$, 则$\exists g\in C(\mathbb{R}^n), \left.g\right|_A=1, \left.g\right|_B=0$.且$g$满足$|g(x)|\leqslant 1,\forall x\in \mathbb{R}^n$. \par
\end{lemma}

\begin{proof}
  可以自然想到距离定义, 于是有$g=\frac{d(x,B)}{d(x,A)+d(x,B)}$.\par
  其中$d(x,A)=\inf_{y\in A}|x-y|$, $d(x,B)=\inf_{y\in B}|x-y|$. 同样可以验证$g(x)$是满足上面的条件的.\par
\end{proof}

于是我们得到下面的定理, 具体说明了连续函数和可测函数的关系:\par

\begin{theorem}
  (Lusin I) 设$E$是一可测集, $f\in \mathcal{M} (E)$, 则$\forall \delta >0, \exists F\subset E, m(E\backslash F)<\delta,$ 其中$F$是闭集, 且$f$在$F$上连续.\par
\end{theorem}

\begin{proof}
  由$f\in \mathcal{M} (E)$有, $\exists \psi_k \rightarrow f$, 其中$\psi_k$是简单函数.\par
  若$f$有界可测, 则$\psi_k$有界, $\exists F_k, m(E\backslash F_k)<\frac{\delta}{2^{k+1}}$, 其中$F_k$是闭集. 此时$\psi_k$在$F_k$上连续.\par
  令$F=\bigcap_{k=1}^{\infty}F_k$, 则$m(E\backslash F)\leqslant \sum_{k=1}^{\infty}m(E\backslash F_k)<\delta$. 于是$f$在$F$上连续.\par
  若$f$无界, 构造$g(x)=\frac{f(x)}{1+|f(x)|}$, 则$g$是有界的, 且$g\in \mathcal{M} (E)$.\par
  于是$\exists F\subset E, m(E\backslash F)<\delta$, 其中$F$是闭集, 且$g$在$F$上连续.\par
  由$g(x)=\frac{f(x)}{1+|f(x)|}$, $g(x)+g(x)|f(x)|=f(x)$, 即$g(x)+|g(x)|f(x)=f(x)$, $f(x)=\frac{g(x)}{1-|g(x)|}$, $f$在$F$上连续.\par
\end{proof}

事实上我们还有$Lusin II$, 可以表示如下:\par

\begin{theorem}
  (Lusin II) 设$E$是一可测集, $f\in \mathcal{M} (E)$, 则$\forall \delta >0, \exists g\in C(\mathbb{R}^n), \exists F\subset E, m(E\backslash F)<\delta,$ 其中$F$是闭集, 且$g(x)=f(x),\forall x \in F$.\par
\end{theorem}

\begin{proof}
  由Lusin I, 我们有$\exists F\subset E, m(E\backslash F)<\delta,$ 其中$F$是闭集, 且$f$在$F$上连续.\par
  令$g=\left.f\right|_{F}$, 则$g\in C(F)$.\par
  为了证明$g\in C(\mathbb{R}^n)$, 我们构造$g_k(x)=\left\{\begin{array}{ll}
    g(x), & x\in F, \\
    \frac{1}{k}, & x\in E\backslash F.
  \end{array}\right.$\par
  则$g_k\in C(\mathbb{R}^n)$, 且$\forall x\in F, g_k(x)=g(x)$, $\forall x\in E\backslash F, g_k(x)=\frac{1}{k}$.\par
  于是$\forall x\in E, g_k(x)\rightarrow g(x)$, 这说明$g_k\rightarrow g$在$E$上收敛.\par
  由于$g_k$是连续的, 因此$g\in C(\mathbb{R}^n)$.\par
\end{proof}
\newpage

\chapter{Lebesgue积分}

\section{非负可测函数的积分}

在讨论一般的积分之前, 我们先定义下方图形, 这个概念实际上是Lebesgue积分的几何意义.\par

\begin{definition}
  设$f$是定义在$E$上的非负可测函数, 则称$f$的下方图形称为$G(E,f)=\{(x,z)\in \mathbb{R}^2: x\in E, 0\leqslant z< f(x)\}$.\par
\end{definition}

\begin{remark}
  这里我们要求$0\leqslant z< f(x)$, 实际上$0\leqslant z\leqslant f(x)$所表达的几何意义是一样的, 但按照后者定义证明函数Lebesgue可测时需要更深层次的分析技术.\par
\end{remark}

下面我们依次定义特征函数和简单函数的Lebesgue积分, 后面我们将用这些函数来定义一般的非负可测函数的Lebesgue积分.\par

\begin{definition}
  设$f(x)=\chi_{A}(x)$, 则$\int_{E}f(x)\mathrm{d}x=1\cdot m(A\cap E)$称为$f(x)$在$E$上的Lebesgue积分.\par
  设$f(x)=\sum_{k=1}^{N}c_k\cdot \chi_{A_k}(x)$, 则$\int_{E}f(x)\mathrm{d}x=\sum_{k=1}^{N}c_k\cdot m(A_k\cap E)$称为$f(x)$在$E$上的Lebesgue积分.\par
\end{definition}

\begin{remark}
  显然可以得到特征函数和简单函数的Lebesgue积分的几何意义就是对应的下方图形的测度.\par
\end{remark}

\begin{lemma}
  设$f$和$g$是定义在$E$上的非负简单函数, 则:\par
  \begin{itemize}
    \item $\int_{E}f(x)\mathrm{d}x\geqslant 0$.
    \item 若$f(x)\leqslant g(x)$, 则$\int_{E}f(x)\mathrm{d}x\leqslant \int_{E}g(x)\mathrm{d}x$.
    \item $\int_{E}(af(x)+bg(x))\mathrm{d}x=a\int_{E}f(x)\mathrm{d}x+b\int_{E}g(x)\mathrm{d}x$, 其中$a,b\in \mathrm{R}^{+}$.
  \end{itemize}
\end{lemma}

下面我们定义一般的非负可测函数的Lebesgue积分.\par

\begin{definition}
  设$f:E\rightarrow [c,d]$是非负可测函数, 则$f$在$E$上有Riemann积分$\lim_{|\lambda|\rightarrow 0}\sum_{k=1}^{N}f(\xi_k)\cdot \Delta x_k$, 类似地, 我们有Lebesgue积分$\lim_{|\lambda|\rightarrow 0}\sum_{k=1}^{N}\eta_k\cdot m(E_k)$.\par
\end{definition}

不难注意到, 这里的$\eta_k$是$f$在$E_k$上的平均值, 而$m(E_k)$是$E_k$的测度, 因此$\eta_k\cdot m(E_k)$实际上是下方图形的面积. 但从严格证明的角度, 这种方式并不方便处理抽象测度的集合, 因此我们考虑下面这种定义方式:\par

\begin{definition}
  设$f:E\rightarrow [c,d]$是非负可测函数, 则$f$在$E$上有Lebesgue积分$\lim_{k\to \infty}\int_{E}\phi_k(x)\mathrm{d}x$, 其中$\phi_k(x)$为单增简单函数, 且$\phi_k(x)\rightarrow  f(x)$, $\forall x\in E$.\par
\end{definition}

这种定义方式看似简单, 但实际上需要证明$\phi_k(x)$的存在性, 并且若出现$\phi_k(x)$和$\psi_k(x)$时, 需要进一步说明$\lim_{k\to \infty}\int_{E}\phi_k(x)\mathrm{d}x=\lim_{k\to \infty}\int_{E}\psi _k(x)\mathrm{d}x$.\par

\begin{definition}
  设$f:E\rightarrow [c,d]$是非负可测函数, 则$f$在$E$上有Lebesgue积分$\sup\left\{\int_{E}\phi(x)\mathrm{d}x: 0\leqslant \phi(x) \leqslant f(x)\right\}$, 其中 $g(x)$是一列非负简单函数.\par
\end{definition}

后面的各种Lebesgue积分的问题中我们都只采用第三种方式定义, 但可以证明三种定义方式时等价的.\par

\begin{lemma}
  设$f$和$g$是定义在$E$上的非负可测函数, 则:\par
  \begin{itemize}
    \item $\int_{E}f(x)\mathrm{d}x\geqslant 0$.
    \item 若$f(x)\leqslant g(x)$, 则$\int_{E}f(x)\mathrm{d}x\leqslant \int_{E}g(x)\mathrm{d}x$.
    \item $\int_{E}(af(x)+bg(x))\mathrm{d}x=a\int_{E}f(x)\mathrm{d}x+b\int_{E}g(x)\mathrm{d}x$, 其中$a,b\in \mathrm{R}^{+}$.
  \end{itemize}
\end{lemma}

\begin{theorem}
  设$f$是定义在$E$上的非负可测函数, 则$\int_{E}f(x)\mathrm{d}x=m(G(E,f))$.\par
\end{theorem}

\begin{proof}
  由$f$是非负可测函数, 则$\exists \phi_k(x)$是单增简单函数, 且$\phi_k(x)\rightarrow f(x)$, $\forall x\in E$.\par
  于是$m(G(E,\phi_k))\leqslant m(G(E,f))$. 于是$\int_{E}f(x)\mathrm{d}x\leqslant m(G(E,f))$. 下证$\int_{E}f(x)\mathrm{d}x\geqslant m(G(E,f))$:\par
  $\forall (x,z)\in G(E,f), x \in E, 0\leqslant z<f(x)$, 则$\exists \phi_{k0}(x)$满足$0\leqslant z<\phi_k(x)<f(x)$, $\forall x\in E$.\par
  即$(x,z)\in G(E,\phi_{k_0})\subset \bigcup_{k=1}^{\infty}G(E,\phi_{k})\subset G(E,f)$.\par
  由$(x,z)$的任意性, 有$m(G(E,f))\subset \bigcup_{k=1}^{\infty}(G(E,\phi_k))$.\par
  于是$m(G(E,f))=\lim_{k\to \infty}m(G(E,\phi_k))$. 即$\int_{E}f(x)\mathrm{d}x\geqslant m(G(E,f))$.\par
\end{proof}

下面围绕非负可测函数的Lebesgue积分, 我们讨论一些性质.\par

\begin{corollary}
  设$f\in \mathcal{M} (E)$, $f\geqslant 0$, 若$\int_{E}f(x)\mathrm{d}x=0$, 则$f(x)=0$, a.e. on $E$.\par
\end{corollary}

\begin{proof}
  假设$m(E[f\ne 0])>0$, 则$m(E[f> 0])>0$, 于是$m(\bigcup_{n=1}^{\infty}E[f>\frac{1}{n}])$.\par
  于是$\exists n_0>0, \delta>0, m(E[f>\frac{1}{n_0}])\geqslant \delta>0$.\par
  此时就会有$\int_{E}f(x)\mathrm{d}x\geqslant \frac{1}{n_0}\cdot m(E[f>\frac{1}{n_0}])\geqslant \frac{\delta}{n_0}>0$.\par 
\end{proof}

\begin{corollary}
  设$f\in \mathcal{M} (E)$, $m(E)\geqslant 0$, $f>0$ a.e. on $E$, 则$\int_{E}f(x)\mathrm{d}x>0$.\par
\end{corollary}

\begin{proof}
  由$f>0$ a.e. on $E$, 则$m(E[f>0])>0$. 于是$\forall \varepsilon >0, m(E[f>\varepsilon])>0$.\par
  于是$\int_{E}f(x)\mathrm{d}x\geqslant \varepsilon\cdot m(E[f>\varepsilon])>0$.\par
\end{proof}

\begin{theorem}
  (Levi) 设${f_n}\in \mathcal{M}(E)$, $f_n(x)\geqslant 0$ a.e. on $E$, 且$f_n(x)$单增收敛至$f(x)$, 则$f\in \mathcal{M}(E)$, 且$\lim_{n \to \infty}\int_{E}f_n(x)\mathrm{d}x= \int_{E}f(x)\mathrm{d}x$.\par 
\end{theorem}

\begin{proof}
  考虑证明$\lim_{n \to \infty}G(E,f_n) = G(E,f)$:\par
  $\lim_{n \to \infty}G(E,f_n) \subset G(E,f)$: $\forall (x,z)\in G(E,f_n), x\in E, 0\leqslant z<f_n(x)$, 则$\exists N>0, \forall n\geqslant N, f_n(x)\leqslant f(x)$, 于是$(x,z)\in G(E,f)$.\par
  $\lim_{n \to \infty}G(E,f_n) \supset G(E,f)$: $\exists f_{n_0}(x)$满足$0\leqslant z<f_{n_0}(x)<f(x)$, $\forall x\in E$, 则$(x,z)\in G(E,f_{n_0})\subset \bigcup_{n=1}^{\infty}G(E,f_n)=\lim_{n \to \infty}G(E,f_n)\subset G(E,f)$.\par
  因此$\lim_{n \to \infty}G(E,f_n) = G(E,f)$.\par
\end{proof}

\begin{corollary}
  (Lebesgue 逐项积分) 设${f_n}\in \mathcal{M}(E)$, $f_n(x)\geqslant 0$ a.e. on $E$, $\forall n$, 则$\int_{E}\left(\sum_{n=1}^{\infty}f_n(x)\right)\mathrm{d}x = \sum_{n=1}^{\infty}\left(\int_{E}f_n(x)\mathrm{d}x\right)$.\par
\end{corollary}

\begin{proof}
  记$P_n(x)=\sum_{k=1}^{n}f_k(x)$, 则$P_n(x)$是单增的, 且$P_n(x)\rightarrow \sum_{k=1}^{\infty}f_k(x)$ a.e. on $E$.\par
  由Levi定理, $\lim_{n \to \infty}\int_{E}P_n(x)\mathrm{d}x= \int_{E}\left(\sum_{k=1}^{\infty}f_k(x)\right)\mathrm{d}x$.\par
  由于$P_n(x)=\sum_{k=1}^{n}f_k(x)$, 则$\int_{E}P_n(x)\mathrm{d}x=\sum_{k=1}^{n}\left(\int_{E}f_k(x)\mathrm{d}x\right)$.\par
  于是$\sum_{k=1}^{\infty}\left(\int_{E}f_k(x)\mathrm{d}x\right)= \int_{E}\left(\sum_{k=1}^{\infty}f_k(x)\right)\mathrm{d}x$.\par
\end{proof}

\begin{theorem}
  (Fatou) 设${f_n}\in \mathcal{M}(E)$, $f_n(x)\geqslant 0$ a.e. on $E$, 则$\int_{E}\liminf_{n \to \infty}f(x)\mathrm{d}x\leqslant \liminf_{n \to \infty}\int_{E}f_n(x)\mathrm{d}x$.\par
\end{theorem}

\begin{proof}
  $\liminf_{n \to \infty}f_n(x)=\lim_{N \to \infty}\inf_{n\geqslant N}f_n(x)$, 记$g_N(x) = \inf_{n\geqslant N}f_n(x)$, 于是$\inf_{n\geqslant N}\int_{E}f_n(x)\mathrm{d}x\geqslant \int_{E}g_N(x)$.\par
  两边取极限有$\lim_{N \to \infty}\inf_{n\geqslant N}\int_{E}f_n(x)\mathrm{d}x\geqslant \int_{E}\left(\liminf_{n \to \infty}f_n(x)\right)\mathrm{d}x$.\par
\end{proof}

\newpage

\section{一般可测函数的积分}

对于一般可测函数, 我们考虑将其分成正部函数和负部函数, 于是有下面的定义:\par

\begin{definition}
  $f$在$E$上的Lebesgue积分定义为$\int_{E}f(x)\mathrm{d}x=\int_{E}f^{+}(x)\mathrm{d}x-\int_{E}f^{-}(x)\mathrm{d}x$.\par
\end{definition}

\begin{lemma}
  考虑$f$在$E$上的积分存在和可积的充要条件, 我们有:\par
  \begin{itemize}
    \item $f$在$E$上积分存在, 即$\int_{E}f^{+}(x)\mathrm{d}x$ 和$\int_{E}f^{-}(x)\mathrm{d}x$至多一个为$\infty$.
    \item $f$在$E$上可积, 即$\int_{E}f^{+}(x)\mathrm{d}x$ 和$\int_{E}f^{-}(x)\mathrm{d}x$均有限.
  \end{itemize}
\end{lemma}

\begin{proof}
  $f$在$E$上积分存在, 即$\int_{E}f(x)\mathrm{d}x=\int_{E}f^{+}(x)\mathrm{d}x-\int_{E}f^{-}(x)\mathrm{d}x$可以运算, 显然需要$f^{+}$和$f^{-}$至少有一个是有限的.\par
  $f$在$E$上可积, 即$|\int_{E}f(x)\mathrm{d}x|<+\infty$, 则$\int_{E}f^{+}(x)\mathrm{d}x$ 和$\int_{E}f^{-}(x)\mathrm{d}x$均有限.\par
\end{proof}

下面给出一些一般可测函数的性质:\par

\begin{proposition}
  \begin{itemize}
    \item 若$f=g$ a.e. on $E$, 则$\int_{E}f(x)\mathrm{d}x=\int_{E}g(x)\mathrm{d}x$.\par
    \item 若$f\in L(E_1)$, $f\in L(E_2)$, 则$f\in L(E_1\cup E_2)$.\par
    \item 若$f\in L(E)$, 则$\left|\int_{E}f(x)\mathrm{d}x\right|\leqslant \int_{E}|f(x)|\mathrm{d}x$.\par
    \item 若$f\in L(E)$, 则$f$在$E$上几乎处处有限.\par
  \end{itemize}
\end{proposition}

\begin{proof}
  \begin{itemize}
    \item 由$f=g$ a.e. on $E$, 则$m(E[f\ne g])=0$, 于是$\int_{E}f(x)\mathrm{d}x=\int_{E}g(x)\mathrm{d}x$, 其中$\int_{E[f\ne g]}f(x)\mathrm{d}x=\int_{E[f\ne g]}g(x)\mathrm{d}x=0$.\par
    \item 由$f\in L(E_1)$, $f\in L(E_2)$, 则$\int_{E_1\cup E_2}f^{+}(x)\mathrm{d}x<\int_{E_1}f^{+}(x)\mathrm{d}x+\int_{E_2}f^{+}(x)\mathrm{d}x<+\infty$, $\int_{E_1\cup E_2}f^{-}(x)\mathrm{d}x<\int_{E_1}f^{-}(x)\mathrm{d}x+\int_{E_2}f^{-}(x)\mathrm{d}x<+\infty$, 于是$\int_{E_1\cup E_2}f(x)\mathrm{d}x<+\infty$.\par
    \item 由$f\in L(E)$, 则$\int_{E}|f(x)|\mathrm{d}x<+\infty$, 于是$\left|\int_{E}f(x)\mathrm{d}x\right|=\left|\int_{E}f^{+}(x)\mathrm{d}x-\int_{E}f^{-}(x)\mathrm{d}x\right|\leqslant \left|\int_{E}f^{+}(x)|\mathrm{d}x\right|+\left|\int_{E}f^{-}(x)|\mathrm{d}x\right|=\int_{E}\left|f(x)\right|\mathrm{d}x$.\par
    \item $f$ a.e. 有限等价于$f^{+}$和$f^{-}$ a.e. 有限. 由$f\in L(E)$, $\int_{E}f^{+}(x)\mathrm{d}x\geqslant \int_{E[f^{+}=\infty]}f^{+}(x)\mathrm{d}x\geqslant N\cdot m(E[f^{+}=\infty]), \forall N \in \mathbb{N}$. 于是$m(E[f^{+}=\infty])<\frac{a}{N}, \forall N\in \mathbb{N}$. 即$m(E[f^{+}=\infty])=0$. $f^{-}(x)$同理可证. 于是$f$在$E$上几乎处处有限.\par
  \end{itemize}
\end{proof}

\begin{remark}
  若$f\in L(E_i), i=1,2,\cdots$, 未必有$f\in L\left(\bigcup_{i=1}^{\infty}E_i\right)$, 也未必有$f$在$\bigcup_{i=1}^{\infty}E_i$上积分存在.\par
  例如$f(x)=1$, $E_i=[i,i+1]$, $\bigcup_{i=1}^{\infty}E_i=[1,\infty)$, 此时$f$在$[1,\infty)$上不可积.\par
  另可取$f(x)=(-1)^{i}$, $E_i=[i,i+1]$, 此时$f$在$\bigcup_{i=1}^{\infty}E_i$上积分不存在.\par
\end{remark}

下面考虑一种方法, 用来控制$f_n$在$E$上的积分值, 为了说明这个特点, 我们首先需要引入绝对连续性, 并结合上面提及过的依测度收敛给出下面一个定理.\par

\begin{definition}
  设$f\in L(E)$, 若$\forall \varepsilon >0, \exists \delta >0, \forall e\subset E, m(e)<\delta$, 有$\left|\int_{e}f(x)\mathrm{d}x\right|<\int_{e}\left|f(x)\right|\mathrm{d}x<\varepsilon$, 则称$f$在$E$上绝对连续.\par
\end{definition}

\begin{theorem}
  设$f\in L(E)$, 则$f$在$E$上绝对连续.\par
\end{theorem}

\begin{proof}
  由$f\in L(E)$, 则$\int_{E}|f(x)|\mathrm{d}x<+\infty$, 下面考虑取${\varphi_k}$单增简单函数逼近$f$, 于是$\int_{E}|f|\mathrm{d}x=\lim_{k \to \infty}\int_{E}\varphi_k(x)\mathrm{d}x$.\par
  即$\forall \varepsilon >0, \exists K>0, \forall k>K, \left|\int_{E}|f|\mathrm{d}x-\int_{E}\varphi_k(x)\mathrm{d}x\right|<\frac{\varepsilon}{2}$.\par
  考虑$f$可积, 于是$|f|=\sum_{k=1}^{n}c_k\chi_{E_k}(x)$, 记$M=\sup \{c_k:k=1,2,\cdots\}\geqslant |f|$, 则有$\int_{e}f(x)\mathrm{d}x\leqslant \int_{e}M\mathrm{d}x\leqslant M\cdot m(e) = M\cdot \delta$, $\forall e\subset E$.
  取$\delta = \frac{\varepsilon}{2M}$, 则$\int_{e}|f(x)|\mathrm{d}x\leqslant \frac{\varepsilon}{2}+\frac{\varepsilon}{2M}\cdot M=\varepsilon$.\par
  于是$ \exists \delta >0, \forall e\subset E, m(e)<\delta$, 有$\left|\int_{e}f(x)\mathrm{d}x\right|<\int_{e}\left|f(x)\right|\mathrm{d}x<\varepsilon$.\par
  这说明$f$在$E$上绝对连续.\par
\end{proof}

\begin{theorem}
  (Lebesgue 控制收敛) 设$f_n$几乎处处有限且可测, $F(x)\geqslant |f_n(x)|$, $f_n\Rightarrow f$在$E$上, 则$f\in L(E)$, 且$\int_{E}f(x)\mathrm{d}x=\lim_{n\to \infty} \int_{E}f_n(x)\mathrm{d}x$.\par
\end{theorem}

\begin{proof}
  由在$E$上$f_n\Rightarrow f$, $\forall \varepsilon >0, \lim_{n \to \infty}m(E[|f_n-f|\geqslant \varepsilon])=0$. 又有
\end{proof}

\newpage

\section{可积函数的连续性}


\newpage

\section{Lebesgue积分与Riemann积分的关系}

\newpage

\section{重积分与累次积分的关系}


\newpage

\chapter{习题集}

\begin{example}
  集合可以与其真子集对等.
\end{example}

\begin{proof}
  在连续集和离散集中均存在例子, 考虑$\mathbb{N}$和$\mathbb{N}\backslash \{1\}$, 则$\mathbb{N}$和$\mathbb{N}\backslash \{1\}$是等势的.\par
  另考虑$\mathbb{R}$和$[0,1]$, 可以构造双射$f:\mathbb{R}\rightarrow [0,1]$, 例如$f(x)=\frac{x}{1+|x|}$, 则$f$是双射, 且$f(\mathbb{R})=[0,1]$.\par
  于是$\mathbb{R}$和$[0,1]$是等势的.\par
\end{proof}

\begin{example}
  稠密集的余集不一定是疏朗集.
\end{example}

\begin{proof}
  考虑$\mathbb{Q}$和$\mathbb{R}\backslash \mathbb{Q}$, 则$\mathbb{Q}$和$\mathbb{R}\backslash \mathbb{Q}$均为稠密集, 但$\mathbb{R}\backslash \mathbb{Q}$的余集$\mathbb{Q}$不是疏朗集.\par
\end{proof}

\begin{example}
  $A \backslash C =  B \backslash C \nRightarrow A = B$.
\end{example}

\begin{proof}
  反例可取$A=\{1,2,3\}, B=\{1,3\}, C=\{2\}$, 则$A \backslash C = \{1\} \nRightarrow B \backslash C = \{1,3\}$. 但$A=B$.\par
\end{proof}

\begin{example}
  全体有理系数多项式是可数集.
\end{example}

\begin{proof}
  设$P_n(x)=a_0+a_1x+\cdots+a_nx^n$, 其中$a_i\in \mathbb{Q}$, 则$P_n(x)$的系数个数为$n+1$, 因此全体有理系数多项式的个数为$\bigcup_{n=0}^{\infty}\mathbb{Q}^{n+1}$.\par
  由于$\mathbb{Q}$是可数集, $\mathbb{Q}^{n+1}$是可数集, 因此$\bigcup_{n=0}^{\infty}\mathbb{Q}^{n+1}$是可数集.\par
\end{proof}

\begin{example}
  开集一定是某一列闭集的并集.
\end{example}

\begin{proof}
  设$E$是开集, 则$\forall x\in E, \exists r>0, B(x,r)\subset E$, 于是$E=\bigcup_{x\in E}B(x,r)$.\par
  由于$B(x,r)$是闭集, 因此$E$是某一列闭集的并集.\par
\end{proof}

\begin{example}
  任意多个可测集的交集不一定是可测集.
\end{example}

\begin{proof}
  只能说明至多可列的情形是成立的.\par
\end{proof}

\backmatter


\end{document}
